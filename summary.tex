

\GenericNonNumberedSection{Summary}{
Clusters of atoms have remarkable optical properties that were exploited since
the antiquity. It was only during the late 20$^{\textrm{th}}$ though that their
production was better controlled and opened the door to a better understanding
of matter. Lasers are the tool of choice to study these nanoscopic objects so
scientists have been blowing clusters with high intensities and short duration
laser pulses to gain insights on the dynamics at the nanoscale.  Clusters of
atoms are an excellent first step in the study of bio-molecules imaging.
New advancements in laser technology in the shape of Free Electron Lasers (FEL)
made shorter and shorter wavelengths accessible from the infrared (IR) to the
vacuum and extreme ultra-violet (VUV and XUV) to even X-rays. Experiments in
these short wavelengths regimes revealed surprisingly high energy absorption
that are yet to be fully explained.

This thesis tries to increase the global knowledge of clusters of rare-gas
atoms interacting with short duration and high intensity lasers in the VUV and
XUV regime. Theoretical and numerical tools were developed and a novel model
of energy transfer based on excited states will be presented.

The first part describes the current knowledge of laser-cluster interaction in
the short wavelength regime followed by the description of the new model. In the
second part of the thesis the different tools and implementations used
throughout this work are presented. Third, a series of journal articles (or
which three are published and one to be submitted) are included where our models
and tools were successfully used to explain experimental results.
}
