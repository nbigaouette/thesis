\section{Appendix: Open-Source Packages}
\label{appendix:oss}

Many different open-source packages were used throughout this work. While not
related to physics \textit{per se}, their value is huge. During the years, I
often found myself recommending some to my colleagues. I thus propose listing
some of the most useful ones that, in my opinion, should be used more often
by scientists.

All the specified packages are available on all platforms (Windows, Linux, Mac
and even BSD) except stated otherwise.


\subsection*{Python}

\textit{Python} is an extremely versatile programming language. Being \textit{interpreted}
(as opposed to \textit{compiled}), it allows fast development with its high
level design. Many programming languages are praised by their user but deeper
investigation sometimes reveals them to be complicated and hard to learn.

Python has code readability as its design philosophy.
A program written in Python can have a single line and can scale extremely
large projects like web sites or video games.

Documentation is of high quality and the community is always helpful. It is
available here: \url{http://docs.python.org/3}

Python is the ideal tool for almost every coding project. The most
important drawback is the fact that it is interpreted. As all interpreted
languages, it is slower than compiled languages because of the overhead of the
interpreter. This only prevents the most performance critical algorithms to be
implemented directly in Python, and different wrappers allow calling compiled
routines (C, C++ or Fortran for example).

The recommended way to install Python on Windows is to use Python(x,y) described
later. On other platforms, package managers make the installation a breeze.


\subsubsection*{NumPy and SciPy}

Fortran and C/C++ users will surely want to work with arrays of data. The closest
to arrays Python can offer is a \textit{list} which is not suited for fast
loops over them.

A fast array implementation for Python is \textit{NumPy}. It includes support
for multi-dimensional arrays, matrices and some high level mathematical functions
that act on these arrays. NumPy allows fast execution of arrays operation by
using, under the hood, compiled routines.

Additionally to NumPy, \textit{SciPy} is the scientific library for Python that
adds support for more specific scientific mathematical functions and algorithms.
It includes for example linear algebra algorithms, discrete Fourier transforms
(FFT), optimization algorithms, statistical routines, integration functions, etc.

NumPy's homepage can be found at \url{http://www.numpy.org} and SciPy at\\
\url{http://www.scipy.org}.


\subsubsection*{Matplotlib}

\textit{Matplotlib}~\cite{matplotlib} is the plotting library for Python. It is extremely powerful
and supports a large range of different figure types (see the gallery
at \url{http://matplotlib.org/gallery.html} for
many examples). Almost all figures in
this thesis were generated by Python and Matplotlib.

It is an easy to library. It's syntax is close to the (proprietary and expensive)
Matlab and thus makes a great replacement for it. The quality of the generated
figures is also, in my opinion, highly superior to Matlab's.
More information on Matplotlib can be found on its homepage at
\url{http://matplotlib.org}.


\subsubsection*{Python(x,y)}

On Linux, it is extremely easy to install and use Python, NumPy, SciPy and
Matplotlib. On Windows though, their homepage gives link to download their
source code which can then be compiled. This is something Windows users are
probably not willing to do.

Instead, I recommend \textit{Python(x,y)} which is a complete distribution of
Python and \textit{close to one hundred} libraries, including NumPy, SciPy,
Matplotlib and many others. It also includes a great Integrated Development
Environment (IDE) called \textit{Spyder}.

With Python(x,y) one can replace fully Matlab on Windows as it includes
everything required for scientific programming.

More information on Python(x,y), including download links, can be found on its
homepage at
\url{https://code.google.com/p/pythonxy/wiki/Welcome}.


\subsection*{NetCDF}

\textit{NetCDF} (Network Common Data Form) is both a set of libraries and a file
formats. It basically allows storing architecture and machine independent data
in a self-describing and array-oriented way.

The self-describing nature means that all information required to interpret the
data is included in the actual file. For example, saving an array representing
particle velocities might contain not only the data, but also vital metadata
like the array size and optional information like data units. This gives
great flexibility to analyze the data as one does not need to worry about
incorrectly interpreting it. NetCDF can also compress the data when saving it,
allowing significant size reduction files output.

NetCDF files are portable and thus can be accessed (read and write) on all
platforms and from many different languages. Wrappers exist for C, C++, Fortran,
Python and others.

NetCDF was created (and is still developed) by the University Corporation for
Atmospheric Research (UCAR) and can be downloaded at \\
\url{http://www.unidata.ucar.edu/software/netcdf/}.


\subsection*{PyMOL}

\textit{PyMOL} is a molecular visualization system written in Python.
It is used here to visualize and animate clusters after simulations. It was
selected over others (like VMD) since it was faster and allowed changing the
number of particles during the simulations, an important requirement.
Figure \ref{fig:md:cluster} was created with PyMOL.

PyMOL can be downloaded for all platforms at \url{http://www.pymol.org}. The
wiki, located at \url{http://pymolwiki.org} contains a great deal of information,
documentation and examples.


\subsection*{Valgrind}

As described in section \ref{section:tools:opencl:valgrind}, \textit{Valgrind}
is an extremely useful tool for development of (compiled) programs. It allows
analyzing the memory access of a program and can reveal many subtle programming
errors. For example, arrays' out-of-range access or using uninitialized
variables are problems that can be quite hard to diagnose but still have
significant consequences.

One of the first tip I can give to someone writing any kind of software is to
run the program through Valgrind. In all but the trivial cases Valgrind
\textit{will} detect problems, even for experienced programmers. It must be
noted that while the MD, QFDTD and all other libraries developed for this work
had errors, Valgrind detected them and I am proud to say that the packages are
now free of \textit{all} of these errors, raising confidence in the simulations
results.

Valgrind is available on Linux and Mac only (unfortunately not on Windows). It's
homepage can be found here: \url{http://www.valgrind.org}.


\subsection*{Git}

\textit{Git} is a version control system (VCS). A VCS allows keeping a history
of some work, for example a program's source code or even a full thesis. Git
has many advantages over other VCS (like \textit{cvs} or \textit{subversion}).
It is \textit{distributed}, meaning it does not depend on a centralized server.
Branches and tags can be created easily, specially compared to subversion,
to explore new ideas.

Git has, unfortunately, a relatively steep learning curve. New users are often
overwhelmed by the vast amount of options or even by its flexibility; Git can
be used in many different workflow.

Keeping a history of one's work is invaluable. Not only it is a good way to keep
track of the development, it can also help debugging. Often, one realize
something broke in the program; what modification introduced the bug? These can
be extremely hard problems to debug. Using a revision control system, it is
possible to pinpoint the exact modification that created the problem.

Git's homepage \url{http://git-scm.com} host its documentation. For Windows
integration, many options exist.
SourceTree (\url{http://www.sourcetreeapp.com/}),
TortoiseGit (\url{https://code.google.com/p/tortoisegit}),
GitExtensions (\url{http://code.google.com/p/gitextensions}
or Git-Cola \\
(\url{http://git-cola.github.io}) are all great tools.


\subsection*{\LaTeX}

This thesis was written with \LaTeX, a powerful document writing and typesetting.

New users often found it hard to learn, or at least to format their document
to their exact requirements. After having wrote this thesis with University of
Ottawa's formatting requirements, I decided to share my configuration.

My \LaTeX class for the University of Ottawa is thus freely accessible on
GitHub at \url{https://github.com/nbigaouette/uottawa_thesis}. I hope it will
be useful to other students and help them use this high quality software.



