\subsection{Clusters of atoms and strong laser fields}

Clusters of atoms have been in use for a longer time than we might normally
think. Gold clusters were already used in ancient Rome for their optical
properties; their size could be controlled to produce different glass colours.
Gustav Mie was the first to theoretically describe the interaction of light
with (gold) spheres in 1908, explaining the light absorption dependence
on cluster size. During the last part of the twentieth century, the discovery of
C$_{60}$ fullerenes marked the real beginning of cluster studies\cite{Reinhard2004}.

As the Roman saw, the size of these nanoparticles is a key property. Techniques to fully
control the size of produced clusters allowed a wide range of studies and
applications. Three main class of cluster creation techniques
exists\cite{Reinhard2004}. First, supersonic jets methods force high pressure
gas through a small nozzle into a vacuum chamber where atoms condensate into
clusters. Second, gas aggregation methods similarly condensate atoms after
their injection into a gas chamber. Lastly, clusters can be created by breaking
up a material surface by either particle collision, laser ablation or high
electric field. These techniques give great control over the created cluster
size.

Along with the recent developments and advances in nanosciences,
the advance of high-power and short duration lasers opened the
door for ground breaking ultra-fast studies. With femtosecond
(1 fs = 10$^{-15}$ s) lasers it became possible to study electron motion,
similarly to a camera flash that captures a moving scene.


The study of rare gas clusters interaction with femtosecond lasers represents a
merging of nano and ultrafast science.
%
Many femtosecond lasers
have a wavelength of 800 nm in the infrared (IR) and cluster have been studied
thoroughly with this ``long'' wavelength. Investigation of energetic
electrons (keV) or highly charged ions (MeV) emission, X-rays production or even
table-top neutron sources\cite{Krainov2007} were performed since the
1990s\cite{Haberland1994,Brabec2009}.


The IR regime sparked a new field of ultrafast physics. By stripping electrons
from their parent ion through tunnel ionization and accelerating them in the
laser's strong electric field, train of pulses of high harmonics can be created,
a process called High Harmonic Generation (HHG).
Created during a single femtosecond pulse, this train of smaller pulses
can reach the attosecond \mbox{(1 as = 10$^{-18}$ s)} duration, an exciting
developing field\cite{Levesque2006}.


Clusters are invaluable tools because they bridge the gap between single atoms
and solids. Their high density allows them to exhibit collective effects similarly
to bulk materials while their small
size increases their surface to volume ratio.
At the smallest range, clusters can be used as models for small molecules and
at the opposite they still present interesting optical properties even at 10,000
atoms\cite{Reinhard2004}.

Additionally to their great size scalability, clusters explosion by-products
after interaction with a strong laser field are accessible, revealing detailed
information about the dynamics. For example, time-of-flight (TOF) mass
spectrometry can reveal the ions charge state spectrum which
is a signature of the amount
of energy absorbed from the laser pulse. This kind of data is not accessible
in the case of laser-bulk interaction since the majority of matter stays
in the solid.

Clusters of atoms can be composed of different elements. Rare gas atoms have
closed outer electronic shells which makes them less prone to chemical
reactions and so don't interact much with each other. Rare-gas clusters are
thus weakly bound by Van der Waals forces and generally form
an icosahedral structure\cite{Martin1996}. The electronic wavefunctions are
more localized around the nucleus than in metal clusters which simplifies the
cluster environment treatment.

A critical characteristic of laser-cluster interaction for the present
work is the fact that clusters can be studied numerically. Full quantum resolution
is not possible even for clusters consisting of a couple of atoms, while on the
other range of the methods spectrum rates equations and (nano)plasma models are
too macroscopic to reflect the large fields gradient present during cluster
dynamics\cite{Fennel2010}. The tool of choice for this problem is Molecular Dynamics
(MD) where trajectories are treated classically; Newton's equations of motion are
solved at each time steps and quantum effects are included using specific rates.

The importance of laser-cluster interaction as an investigative tool
can be seen by the vast amount of work on the subject, mainly
in the IR regime\cite{Fennel2010}. Varin \textit{et al.}\cite{Varin2012}
recently developed an electrodynamic
particle-in-cell (PIC) code for IR studies on pre-ionized clusters. PIC methods
have a better scaling than MD in terms of particles number and since they treat
the electromagnetic field dynamically (through Maxwell's equations) field
retardation effects are taken into account, which can be important for large
clusters. On the downside PIC methods don't have the precision of MD for close
range interactions and collisions. A novel addition by Varin \textit{et al.}
is to add microscopic corrections to PIC for more realistic close range
interactions\cite{Peltz2012}.
Since the grid must resolve the electromagnetic wave,
PIC simulations can thus be a challenge when the wavelength is small
but this opens the door to a new frontier of modelling.

The interest of this thesis is in modelling short wavelength interaction with
clusters. In recent years, laser light sources have been moving to shorter
wavelengths due to groundbreaking new Free Electron Lasers (FEL)
facilities; Vacuum Ultra-Violet (VUV), Extreme Ultra-Violet (XUV), soft X-Rays
and even hard X-rays are now accessible. In FELs, relativistic electrons are sent
through an undulator in which they emit a coherent pulse, tunable in wavelength
(by changing the electrons initial energy)
from microwaves to X-rays\cite{Brabec2009,Ackermann2007a,Pellegrini2012} at
unprecedented intensities. A single 10 to 100 fs FEL pulse can contain 10$^{13}$
photons, the same amount produced by the best synchrotrons during
\textit{one second}\cite{Bostedt2009}. Even though FEL installations require
large facilities, some researchers are working on a smaller version, as
``short'' as 55 meters\cite{Shintake2008}.

% http://ieeexplore.ieee.org/stamp/stamp.jsp?tp=&arnumber=5500378
At shorter wavelengths, treating the laser as simply an electromagnetic field
during ionization is not valid anymore and photons must be considered instead.
The Keldysh parameter $\gamma$ dictates which ionization regime must be considered. In
cases where $\gamma \ll 1$ the electric field of the laser is strong enough that
the ion's Coulomb potential is distorted to such an extent
that electrons can tunnel out.
However, when $\gamma \gg 1$ the laser frequency is too large compared to the
field's strength. Ionization is then dominated by single (or a few) photon
absorption.
Experiments in
2002\cite{Wabnitz2002,Bostedt2009} at
DESY's FLASH (\textbf{F}ree-electron-\textbf{Las}er in \textbf{H}amburg)
on xenon clusters in the VUV regime (98~nm wavelength, 12.65~eV photon energy)
had $\gamma$ values between 8 and
100, putting it in the regime dominated by photon processes.
During these experiments, unexpected high charge states were seen that could not
be explained by traditional energy transfer mechanisms. Theoretical work was
then performed for explaining these high charge states; these models will be
covered in chapter \ref{section:intro:mechanisms}.



At shorter wavelength,
argon clusters were studied in the XUV regime\cite{Bostedt2008}. The 32.8~nm
(37.8~eV) photons are 22~eV above argon's first ionization
potential and 10.2~eV  above the second. It was found that ionization is a multistep process of
photo-electrons emission and because of the subsequent charge buildup in the
cluster, the electron energy distribution is non-thermal.

In 2006, a proof-of-principle experiment showed that it is possible to
do (soft) X-rays diffraction imaging using FLASH pulses\cite{Chapman2006}:
Chapman \textit{et al.} were able to image a micrometer-wide stick-figure pattern
engraved on a 20 nm thin film. Imaging was possible even though the sample was
eventually destroyed by the high intensity X-ray laser pulse.
Similar studies on xenon clusters were performed in 2010\cite{Bostedt2010} where
diffraction patterns were obtained for clusters in single laser shots
at 13 nm (95.37 eV).

More multiphoton
ionization experiments on nitrogen, argon, neon and helium were performed at
FLASH.
% [8] A.A. Sorokin et al., “Multi-photon ionization of molecular nitrogen by femtosecond soft X-ray FEL pulses,” J. Phys. B: At. Mol. Opt.
% Phys. 39, L299-L304 (2006).
% [9] A. Fohlisch et al., “High-brilliance free-electron-laser photoionization of N2: Ground-state depletion and radiation-field-induced modifi-
% cations”, Phys. Rev. A 76, 013411 (2007).
% [10] R. Moshammer et al., “Few-Photon Multiple Ionization of Ne and Ar by Strong Free-Electron-Laser Pulses,” Phys. Rev. Lett. 98,
% 203001 (2007).
% [11] A. Rudenko et al., “Recoil-Ion Momentum Distributions for Two-Photon Double Ionization of He and Ne by 44 eV Free-Electron
% Laser Radiation,” Phys. Rev. Lett. 101, 073003 (2008).
% [12] A.A. Sorokin et al., “X-ray-laser interaction with matter and the role of multiphoton ionization: Free-electron-laser studies on neon and
% helium” Phys. Rev. A 75, 051402(R) (2007).
% [13] M. Nagasano et al., “Resonant two-photon absorption of extreme-ultraviolet free-electron-laser radiation in helium,” Phys. Rev. A 75,
% 051406(R) (2007).
For the even shorter wavelength of 13.7 nm (90.5~eV), xenon clusters irradiated at
$\ten{7.5}{15}$~W/cm$^2$ produced charged states of up to 21+\cite{Sorokin2007,Richter2009},
meaning 60 XUV photons had to be absorbed per atom. Xenon's giant 4d
resonance is suspected to be the cause of these high charge states.

The Linac Coherent Light Source (LCLS) at SLAC, Stanford, is another important
FEL facility where the first FEL hard X-rays lasing was
performed in 2010\cite{Emma2010,Schneider2010}. Single Neon atoms were studied in
this hard X-rays source\cite{Young2010} showing fully stripped neon nucleus.
Later, protein nanocrystallography was performed
on photosystem~I in 2011\cite{Chapman2011} and on photosystem~II in
2013\cite{Kern2013}. These two large cell membrane proteins are implicated in
photosynthesis of algae, plants, and some bacteria.
By using pulses shorter
than the explosion time scale, the protein's structure could be reconstructed,
opening the door for determination of the biomolecules structure that
do not crystallize correctly for use in traditional crystallography studies.
In the same Nature issue, Mimivirus, the second largest known virus at 450 nm
of diameter, was imaged using single shot hard X-ray pulse (0.69~nm,
1.8~keV)\cite{Seibert2011}. LCLS pulses have a duration of 70 to 500 fs at
wavelengths of 0.15 nm to 2.2 nm\cite{Pellegrini2011}.

Cluster studies continue to this day in regimes such as VUV\cite{Arbeiter2011},
XUV\cite{Murphy2008a,Murphy2008b,Krikunova2012} and X-rays\cite{Ziaja2009b,Thomas2012,Timneanu2013}
and there is much theoretical work to be done as new phenomena are uncovered.

% http://photon-science.desy.de/facilities/flash/publications/selected_publications/index_eng.html
% https://portal.slac.stanford.edu/sites/lcls_public/Pages/Publications.aspx

