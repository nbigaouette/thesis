\chapter{Introduction}
\markright{Introduction} % new right header
%======================================================================
\section{What is \LaTeX ?}
%======================================================================
\LaTeX\index{\LaTeX} is a program for formatting or ``typesetting'' documents. 
Actually, when we talk about \LaTeX\, we are usually referring not only to the 
\program{latex} program, but also to a suite of other utility programs which 
work together to format large, complicated documents or books.

\LaTeX\ is unlike common word processing programs found on micro computers.
It is more like a programming language.
``Source'' files are prepared with a text editor which are then processed by the \program{latex} program to produce the final document.
Tags in the source file tell the \program{latex} program how to format the document.
Also, like a programming language, you can (and should) create your own formatting tags, which are included at the beginning of your source file.

\LaTeX\ is a set of macros written by Leslie Lamport \cite{lamport.book} in the \TeX\ language developed by Donald Knuth \cite{knuth.book}.
You may encounter two versions of \LaTeX .
The ``old'' version is 2.09.
In 1993, a newer version, \LaTeXe , was developed to standardize the many extensions which have been developed to improve functionality.
To distinguish between versions, in \LaTeXe\  the first command in a source document was changed from 
\verb=\documentstyle= 
% \verb command is unusual: delimited by any 2 characters; formatted in-line
to 
\verb=\documentclass=
, and the extension ``packages'' are handled differently (with the new 
\verb=\usepackage=
command).
The standardized packages in \LaTeXe\  are described in a handy companion volume \cite{goossens.book} to Lamport's ``\LaTeX\ Book''\cite{lamport.book}.
These last two books are required reading for serious \LaTeX ing, although often an example document (such as this one) is equally valuable.
%======================================================================
\section{Why Use \LaTeX ?}
%======================================================================
Why use \LaTeX\ rather than a word processing package?
\begin{itemize}
\item The source document is plain (ASCII) text, so is very portable.
Any editor or word processor on any computer can be used to write the source document.
\item \LaTeX\ is available on most computing platforms (the source code is free).
\item It is used by many (especially scientific) publishers to format journals and books.
\item \LaTeX\ takes care of most of the formatting of the document with standard environments based on good typesetting form, letting the author concentrate on the \emph{content}.
\item It is flexible enough to be modified to suit any special needs.
\item \LaTeX\ \emph{easily} keeps track of the numbering and organization of structures in large documents, allowing easy modifications.
\item Large documents can be developed in modular form (chapters or sections in separate files).
\item \LaTeX\ has a superior mathematics formating language with every conceivable mathematical symbol.
\end{itemize}
%======================================================================
\section{Possible Reasons Not To Use \LaTeX}
%======================================================================

\begin{itemize}
\item The software is free --- but you \emph{need} the reference books.
\item There is little on-line help --- although the Web has some information.
\item Error messages can be cryptic, and errors difficult to isolate.
\item You need to constantly be \program{latex}-ing and viewing your source to see what you've done.
\item Word processors, such as WordPerfect, now do a reasonable job of handling large documents (but they're still not as good as \LaTeX ).
\end{itemize}
%======================================================================
\section{Where to Find \LaTeX ?}
%======================================================================
\LaTeX\index{\LaTeX!availability} is available on all Unix hosts on campus.
Since the source is freeware, implementations have been developed for just
about all computing platforms. Some versions are commercial because they provide
additional user interface features such as editors which process the document
in the background, or graphical equation editors. One such commercial product,
Scientific Workplace, is supported at UW\@. It provides a graphical front end to \LaTeX\ in which the user is shielded from the language syntax. See
\texttt{http://ist.uwaterloo.ca/ew/software/scientific/} for details.

\LaTeX\ is not currently on the Waterloo Polaris network.

Also, the Comprehensive TeX Archive Network (CTAN) on the Internet, contains all
 source files and utilities for \LaTeX\ if you wish to download a copy for your
home computer.
