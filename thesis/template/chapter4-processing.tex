\chapter{Processing Your \LaTeX\ Source Files Under Unix}
\markright{Processing Your \LaTeX\ Source Files Under Unix} % new right header
%======================================================================
\section{Running \program{latex}}
%======================================================================
The \program{latex}\index{\LaTeX!running} program is executed on your master source file:
\begin{verbatim}
% latex master  (for a source file called master.tex)
\end{verbatim}
If you are using \program{bibtex} or \program{makeindex}, run these next:
\begin{verbatim}
% bibtex master  (for source file master.tex)
% makeindex master
\end{verbatim}
Run \program{latex} at least twice more to process all the labels and references.
The resulting typeset document is in a file called \verb=master.dvi=.
%======================================================================
\section{Syntax Checking}
%======================================================================
\index{\LaTeX!syntax checking}
You will notice that \LaTeX\ doesn't always give the most instructive error messages.
It is useful to use a utility called \program{lacheck} to check for syntax errors before using the \program{latex} program.
\begin{verbatim}
% lacheck *.tex  (check all your .tex files)
\end{verbatim}
%======================================================================
\section{Spell Checking}
%======================================================================
The Unix \program{correct} program is useful for checking spelling.
\begin{verbatim}
% correct -l master.tex  (-l option filters LaTeX commands)
\end{verbatim}
%======================================================================
\section{Previewing the Typeset Document}
%======================================================================
\index{\LaTeX!previewing}
The typeset document is in a \exten{DVI} file, \eg\ \verb=master.dvi=.
This file may be previewed using either the \program{xdvi} or \program{ghostview} programs.
It is a good idea to view your document frequently as you develop it.
It's easier to debug your \LaTeX\ source files as you work, rather than all at once.

To use \program{xdvi}, run it on the \exten{DVI} file.
\begin{verbatim}
% xdvi master.dvi  (for source file master.tex)
\end{verbatim}

To use \program{ghostview}, process the \exten{DVI} file with \program{dvips} to produce a PostScript file then use \program{ghostview} to view it.
\begin{verbatim}
% dvips master.dvi > master.ps   (for source file master.tex)
% ghostview master.ps
\end{verbatim}
%======================================================================
\section{Printing}
%======================================================================
Once you are satisfied with the final document you may print it using the \program{lpr} command.
\begin{verbatim}
% lpr -Plw -Fd master.dvi  (for source file master.tex)
OR ...
% lpr -Plw master.ps
\end{verbatim}
You may also print directly from \program{ghostview}.

If you wish to print only some of the pages of your document, \program{ghostview} will let you do that.
Alternatively, you can use the \program{dviselect} to print selected pages from a \exten{DVI} file, for example
\begin{verbatim}
% dviselect -i master.dvi -o somepages.dvi 5-12 13 17
\end{verbatim}
will place pages 5--12, 13, and 17 in a new \exten{DVI} file called ``somepages'', which then must be sent to the printer as outlined above.
%======================================================================
\section{Creating Portable Document Format (PDF)}
%======================================================================
If you wish to create an electronic version of your document,
rather than a paper version, a good choice is to create a PDF file, and \LaTeX\ provides utilities for doing so.

PDF documents allow enriched features, such as hyperlinks. The \program{hyperref} package can be included in your master source file to automate internal hyperlinking in your document, as well as other features.

One important point to consider when creating a PDF file is the use of scalable fonts and graphics, to a avoid a ``jaggy'' appearance (loss of resolution) when you magnify a page. 

This course demonstrates the inclusion of Encapsulated Postscript (EPS) files as figures. These are created by ``printing to file'' through a Postscript printer driver from a drawing or plotting application. EPS files are scalable graphics, so work well when creating PDF.

By default, \LaTeX\ uses its own bit-mapped fonts (the Computer Modern set) when a printed document is produced. These bit-mapped fonts create blurry, ``jaggy'' results when converted to PDF. The way around this is to use the Postscript fonts made available under \LaTeXe. Alternative Postscript fonts can be specified in the master document preamble by including font packages. There are also Postscript versions of the default Computer Modern font sets. To make sure the PS versions of the default Computer Modern fonts are used when creating a PS file via \program{dvips}, use the \texttt{-Ppdf} option (see below).

There are two methods to create a PDF file. 
\begin{enumerate}
\item Create PDF directly from your \texttt{.tex} source filesby using the \program{pdflatex} program instead of \program{latex}.
\item Create a Postscript file then use \program{Adobe Acrobat} ``distiller'' or \program{Ghostview/GSView} to create the PDF file.
\begin{verbatim}
% dvips -Ppdf master.dvi > master.ps   (note use of -Ppdf option)
\end{verbatim}
\end{enumerate}
