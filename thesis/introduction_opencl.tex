\subsubsection{Acceleration through OpenCL and GP-GPUs}
A new trend lately in High Performance Computing (HPC) is code acceleration
through graphics cards. Similar to the ubiquitous Moore's Law in the CPU world,
video cards power evolved exponentially during the last twenty years, pushed
by the never ending need of more realistic video games. From ``dumb'' devices
drawing primitive shapes on a display, they evolved to extremely powerful
devices and started to act more like CPU by being programmable (shaders) during
the first half of 2000. The term \textit{Graphical Processing Unit} (GPU) was
coined in 1999 by Nvidia, the biggest vendor of video cards, as a selling point
to their product and to emphasize the fact that their chips were becoming more
and more alike CPUs.

Due to this increase in power, people tried to use these video cards as
accelerators for other things than real video operations. Because GPUs have
intrinsically high parallelism (for example the same operation is performed on
all pixels of an image in parallel) HPC users and developers got interested.
In his 2005 book, Taflove described a way to use a video card's GPU to
accelerate his electrodynamics FDTD solver \cite{Taflove2005}. At that time no
General-Programming GPU (GP-GPU) framework existed so Taflove (and anyone
interested in using GPUs at that time) had to ``translate'' the FDTD algorithm
into one that could be understood by a video card. This process was extremely
hard as it required using low level graphic primitives; in his case, OpenGL
calls. Normally OpenGL is used to draw animated scenes on the user's screen.
Notable examples of OpenGL usage is video games where the user is immersed in a
virtual three dimensional space. The process of re-writing an algorithm into
OpenGL calls is one that only a few highly skilled and knowledgeable people can
tackle.

In 2007, Nvidia saw an opportunity for market expansion. Why not let
non-videogames programmer use the powerful GPUs for something else then video
operations? For this to happen, a programming framework had to be released; the
amount of programmers and scientists able to exploit OpenGL to their advantage
was limited. They thus released their \textit{Compute Unified Device
Architecture} (CUDA), later just renamed CUDA. CUDA allows writing normal C or
C++ programs with some extensions, called \textit{kernels}, that can be run on
the GPU. These kernels have the same structure as C functions but are executed
concurrently by every cores on the GPU\footnote{Technically, GPUs don't have
``cores'' \textit{per se} like CPUs but the comparison can still be made.}. The
number of cores on a recent consumer grade video card is now of the order of
many hundreds; a 50\$ Nvidia GT 620 has 96 CUDA cores and 1 GB of RAM, while the
GTX TITAN has 2688 CUDA cores and 6 GB but cost more than 1,000 \$. Even though
the highest priced GPU can be more expensive than complete workstations, no CPUs
can offer close to three thousand cores for a still affordable price.

A year after CUDA was released, Apple wanted a framework that would allow
programs to be accelerated on their top-of-the-line product's GPU while still
being able to run on their lower-end range of products (which did not have a
discrete video card). Additionally, some Apple products were released with ATI
video cards (Nvidia's main competitor) which, understandably, never supported
CUDA. They thus released a framework called \textit{OpenCL} (standing for
Open Compute Language) in 2008 with the help of many partners (IBM, ATI, Intel
and even Nvidia). While conceptually similar to CUDA (smaller routines are
written in a kernel function and launched individually on the GPUs by the main
program) OpenCL has the advantage of targeting heterogeneous platforms
consisting of (possibly and not limited to) many CPUs, many cores and GPUs. The
main advantage of OpenCL is its portability; a program written in OpenCL can
not only run on GPUs from ATI and Nvidia but also on traditional CPUs,
exploiting all cores available on the CPUs transparently.

It was thus decided to port the two codes (MD and QFDTD) to OpenCL to take
advantage of the GPU power while retaining portability.

Some things are important to consider when porting codes to GPU frameworks.
First, because a core on a GPU is quite slower than a core on a CPU,
parallelisms must be extracted from algorithms. Second, the main drawback of
GP-GPU programming is the fact that GPUs have their own memory, independent of
the system's memory; kernels will only have access to the device's memory. Data
required for kernel calculation must thus be first transferred to the device's
memory and similarly the resulting data must be transferred back to the host
memory where it can be further processed by the rest of the main program. Video
cards today are connected to a computer through PCI-Express (PCIe) connections.
While fast, it can still be a bottleneck if data is to be transferred back and
forth similarly to main memory. It is thus necessary to reduce to a minimum the
data transfer between the host and the device, similarly to communications in a
distributed memory parallel programming paradigm.

In the case of the MD algorithm, every interaction pair is independent of all
others and can thus be calculated concurrently; this is thus the basis of the
OpenCL implementation. The main loop that calculates the electrostatic field
and potential at every particle's position is implemented as an OpenCL kernel.
Each thread on the GPU will thus calculate the interaction between one particle
and all others. Once the MD kernels are launched on the GPU, they are only
halted when either ionization is to be calculated (at each femtoseconds) or
when data needs to be saved (to take a snapshot of the simulation for example).
This prevent the GPU from being interrupted too often and reduces the amount of
data transfers. Using this OpenCL implementation, MD simulations could be run
80 times faster than on conventional CPUs.

As for the QFDTD, only the real time algorithm was implemented as OpenCL
kernels since the imaginary time method did not require long simulation times.
As in the case of the MD, the real time algorithm is left running on the GPU
until data needs to be saved to a snapshot or post-processed to reduce
transfers.

