\section{Introduction}

% See https://wiki.cphoton.science.uottawa.ca/lr/doku.php?id=members:nbigaouette:labbook:thesis:outline

\subsection{General objectives}
% ~5 pages. “Big picture”. Same as detailed, but simplified. The goal is to let
% the reader understands where I am going, why I'm doing it, before he starts
% reading. This should allow him to grasp my explanations more easily. General
% litterature search goes here.


Our capacity control atoms and molecules, specifically with light,
will drive the evolution of nanosciences. The method of choice to study 
bio-molecules today is X-ray crystallography, a method that lead Watson and 
Crick on their discovery of DNA's double-helix structure in 1953. 
Unfortunately, only a small fraction of bio-molecules can be crystallized 
efficiently. Additionally, obtaining large amounts of these molecules can be 
challenging in itself. Could it be possible to obtain a diffraction pattern, 
the basis of X-ray crystallography, with only a single one of them? Could the 
measurement be made directly in water droplets as to prevent bio-molecules' 
denaturation during crystallization? Because large amounts of atoms are 
required for normal X-ray crystallography experiments, the X-ray source will 
have to be many times more energetic to get interesting contrast in the 
case of single-molecule imaging. Since X-rays are ionizing, increasing their 
total energy will have unfortunate consequences for the molecule. This 
light-matter interaction will thus set an upper range on the light duration.

Today's X-ray sources such as synchrotrons don't have the intensity and
duration required for such experiments but new Free-Electron Laser (FEL) are
able to fill the gap. By sending relativistic electrons through an ondulator,
one can efficiently control the wavelength of the resulting laser pulse, from
microwave to X-rays. FEL installations at the Linac Coherent Light Source
(LCLS) in Stanford and Free-Electron Laser in Hamburg (FLASH) have generated
ultra-short (femtosecond) laser pulses at ultra-short wavelength and
unprecedented intensities. By concentrating the energy during a short pulse,
the light-matter interaction time scale is now on the same order as the
electrons movement, allowing such pulses to act as the equivalent of a camera
flash. Such a useful tool can thus be used for probing of molecular
and, more importantly, electronic dynamics. Multiple images could be assembled
to reconstruct a molecular movie; studying how a protein is folding has great
potential for new drug treatment for example.

Light and matter interaction is well understood at the two size extremes:
single atoms and solid systems. Intermediate systems, like nano-particles and
bio-molecules, offer greater challenges. To study these, clusters of atoms are
a useful tool. They bridge the gap between single atoms and solid state
matter. They are nanoscale objects, ranging from less than ten atoms to tens of
thousands of atoms or even more. Even though they are small in size, their
density is close to solid matter and can thus exhibit the same collective
effects that are present in large scale systems. Additionally, they are easily
accessible experimentally; simply forcing a gas through a small enough nozzle
will ``stick'' atoms together and clusters will emerge from the nozzle's pin
hole. Varying the gas pressure will result in different cluster sizes
and they will normally have a structure in icosahedral shells.
Different combinations of atoms are also possible, for example getting a core of
one kind of atom surrounded by shells of another element. More importantly for
the following work is the fact that clusters of atoms are small enough to be
studied numerically. Complexifying a system by adding more particles will
result in extra computational burden, so smaller systems are obviously easier
to simulate. Solving quantum mechanically an atom with many electrons is an
extremely hard problem. A system composed of dozens of atom is thus intractable
without using approximations. Keeping the system as small as possible allows to
reduce to a maximum the number of approximations and stay as close as possible
to the physical system.

Another interesting aspect of clusters is the relative importance of
their boundaries compared to their volume. Because the surface of a sphere
scales as radius squared and the volume as radius cubed, smaller objects can
exhibit more boundary effects.

Experiments with clusters have shown interestingly large energy absorption
compared with gas. High energetic electrons (tens of keV) have been observed.
Ions can also be accelerated to MeV energies and X-rays can be generated by the
electrons' movement around ions in the cluster.

This brings important questions; do we understand fully how the energy is
absorbed by the clusters from the laser? How is that energy distributed
inside a nanoparticle? What are the different mechanisms that play a role in
the interaction? What is the effect and importance of these mechanisms? These
are the kind of questions theoretical studies can answer.

Certain heating mechanisms are known. Three of them are of great importance.
First, single- or multi-photon ionization is the absorption by a bound electron
of one or more photons which promotes this electron to the continuum. This
mechanism is dependent on the photon density; higher intensities mean more
photons which in turns increases the probability of an electron absorbing one
(or more) photon. The resulting electron will have, as its kinetic energy, the
remaining between the ionization potential energy and the photon energy. This
extra kinetic energy is clearly visible in electron spectra obtained from
experiments.



Theoretical investigations are essentials to open the door to a better 
understanding of ultra-short and ultra-intense laser energy absorption by 
nanoscale objects.


