\section{Introduction}





\subsection{General Introduction}
TODO --> same as detailed, but simplified.
The goal is to let the reader understands where I am going, why I'm doing
it, before he starts reading. This should allow him to grasp my explanations
more easily.


\subsection{Detailed Introduction}

\begin{enumerate}
\item Study and control atoms and molecules using light
\item Biomolecules imaging
    \begin{enumerate}
        \item X-Ray crystallography failing for many biomolecules, proteins
        \item Short wavelength (X-rays) required
        \item High energy absorption: single shot required
    \end{enumerate}
\item Ultra short length-scales and ultra fast time-scales
\item Dynamics probing: molecular and electronic
\item $\rightarrow$ Molecular movie
\item Light-Matter
    \begin{enumerate}
        \item Extremes scales well understood (solid vs. single atom)
        \item What about in-between? ``Intermediate'' systems: nanoparticles
    \end{enumerate}
\item Wavelength
    \begin{enumerate}
        \item lot of work @ 800 nm
        \item HHG after tunnel ionization
        \item FEL allows much shorted wavelengths, where less work has been done
        \item Recent studies @ 100 nm and 32 nm $\rightarrow$ surprising results
    \end{enumerate}
\item Cluster studies
    \begin{enumerate}
        \item Nano-scale objects
        \item Solid density
        \item Experimentally accessible
        \item Small system (numerically solvable)
        \item Boundary effects
        \item Absorb energy more efficiently than gas. Emission of:
        \begin{enumerate}
            \item Highly energetic electrons (~10 keV)
            \item Highly charged ions (MeV ?)
            \item X-Rays
            \item Fusion? Neutrons?
        \end{enumerate}
    \end{enumerate}
\item Is energy absorption well understood?
\item Current heating mechanisms
    \begin{enumerate}
    \item Single photon ionization
    \item Impact ionization
    \item Tunnel ionization
    \end{enumerate}
\item Proposed new heating mechanisms:
    \begin{enumerate}
    \item Barrier lowering (for single photon)
    \item Atomic potential
    \item MBR
    \item Impact excitation (ACI)
    \end{enumerate}
\end{enumerate}






% \subsection{Goals}
% \begin{enumerate}
% \item Study the influence of these on the cluster dynamics, energy absorption and distribution
% \begin{enumerate}
%     \item laser field
%     \item cluster environment
%     \end{enumerate}
% \item Introduce a simple new mechanism (ACI)
% \end{enumerate}
% 
% 
% 
% 
% \subsection{Tools}
% \begin{enumerate}
%     \item Molecular Dynamics (MD)
%         \begin{enumerate}
%         \item Classical approximation to particles dynamics
%         \item Tracking of every particles in the system
%         \item Quantum effects added as cross-sections
%             \begin{enumerate}
%             \item Single photon ionization
%             \item Impact ionization
%             \item Impact excitation
%             \item Tunnel (possible, not used)
%             \end{enumerate}
%         \item Interaction of every pair of particles is calculated
%             \item Scales as O(N2)
%                 \begin{enumerate}
%                 \item Good for small clusters
%                 \item Intractable for larger clusters
%                 \end{enumerate}
%             \item Acceleration through OpenCL and GP-GPUs
%         \end{enumerate}
%     \item Quantum FDTD
%         \begin{enumerate}
%         \item Study effect of a laser field and cluster environment on:
%             \begin{enumerate}
%             \item Ionization
%             \item Cross-sections
%             \end{enumerate}
%         \item Grid-based
%             \begin{enumerate}
%             \item Scales as O(N3)
%             \item High precision required close to atoms –> Non-linear mapping
%             \item Acceleration through OpenCL and GP-GPUs
%             \end{enumerate}
%         \end{enumerate}
% \end{enumerate}
% 
% 
% 
% \subsection{Story of the thesis and papers}
% \begin{enumerate}
%     \item Writing of MD code and ionization library from scratch
%     \begin{enumerate}
%         \item Why from scratch? we had specific needs:
%         \begin{enumerate}
%             \item New particles created
%             \item Ionization routines
%             \item Previous tools (treecode) not satisfactory
%         \end{enumerate}
%     \end{enumerate}
%     \item Develop ACI model and its implementation
%     \item Study ACI effect at XUV regime with Argon (matching experiments)
%     \begin{enumerate}
%         \item First paper
%     \end{enumerate}
%     \item Study cluster size effect at XUV regime with Argon (matching experiments)
%     \begin{enumerate}
%         \item Second paper
%     \end{enumerate}
%     \item Study the quantum states in a cluster environment: is ACI a valid model?
%     \begin{enumerate}
%         \item We needed a quantum solver to look at states
%         \item Past experience with FDTD showed great potential
%         \item FDTD easy to understand, implement and parallelize
%         \item Third paper (QFDTD)
%     \end{enumerate}
%     \item Applications of ACI to a different regime; is it still valid?
%     \begin{enumerate}
%         \item Fourth paper (100 nm)
%     \end{enumerate}
% \end{enumerate}
% 
% 
% 
% 
% 
% 
% \subsection{Contributions}
% Contributions of Collaborators and/or Co-Authors
% 
% Clearly distinguishes the contributions of the student from those of all other collaborators or co-authors, and 
% identifies in detail all other contributions.
% 
% \begin{enumerate}
% \item MD from scratch
% \item QFDTD from scratch
% \item Library organization
% \end{enumerate}





\ReferencesSubsection{references}
