\section{Introduction}

% See https://wiki.cphoton.science.uottawa.ca/lr/doku.php?id=members:nbigaouette:labbook:thesis:outline

\subsection{General objectives}
% ~5 pages. “Big picture”. Same as detailed, but simplified. The goal is to let
% the reader understands where I am going, why I'm doing it, before he starts
% reading. This should allow him to grasp my explanations more easily. General
% litterature search goes here.


NOTES:
\begin{itemize}
\item State what's new, what's \underline{exciting}.
\item Don't go into too much details. Cite/reference instead. Just to say that
I am aware of ``it'', but just mention it; don't discuss it.
\item 800 nm $\rightarrow$ saturated
\end{itemize}

Our capacity to control atoms and molecules, specifically with light,
will drive the evolution of nanosciences. The method of choice to study
structures at the nanoscale today is X-ray crystallography, a method that lead
Watson and Crick on their discovery of DNA's double-helix structure in 1953.
Unfortunately, only a small fraction of bio-molecules can be crystallized
efficiently\citeneeded. Additionally, obtaining large amounts of these
molecules can be challenging in itself\cite{Young2010}. Could it be
possible to obtain a diffraction pattern,
the basis of X-ray crystallography, with only a single  molecule? Could the
measurement be made directly in water droplets as to prevent bio-molecules'
denaturation during crystallization? Because large amounts of atoms are
required for normal X-ray crystallography experiments, the X-ray source will
have to be many times more energetic to get interesting contrast in the
case of single-molecule imaging. Since X-rays are ionizing, increasing their
total energy will have unfortunate consequences for the molecule. This
extreme light-matter interaction will thus set an upper range on the X-ray
pulse duration.

Today's main X-ray sources such as synchrotrons don't have the intensity and
duration required for such experiments but new Free-Electron Laser (FEL) are
able to fill the gap. By sending relativistic electrons through an ondulator,
one can efficiently control the wavelength of the resulting laser pulse, from
microwave to X-rays\cite{Ackermann2007a}. FEL installations at the Linac
Coherent Light Source (LCLS) in Stanford and Free-Electron Laser in Hamburg
(FLASH) have generated ultra-short (femtosecond) laser pulses at ultra-short
wavelength and unprecedented intensities. By concentrating the energy during a
short pulse, the light-matter interaction time scale is now on the same order
as the electrons movement, allowing such pulses to act as the equivalent of a
camera flash. Such a useful tool can thus be used for probing of molecular
and, more importantly, electronic dynamics\cite{Chapman2011}. Multiple
images could be assembled to reconstruct a molecular movie; studying how a
protein is folding has great potential for new drug treatment for
example\citeneeded.

Light and matter interaction is well understood at the two size extremes:
single atoms and solid systems. Intermediate systems, like nano-particles,
bio-molecules or even viruses\cite{Seibert2011}, offer greater challenges. To
study these, clusters of atoms are a useful tool. They bridge the gap between
single atoms and solid state matter. They are nanoscale objects, ranging from
less than ten atoms to tens of thousands of atoms or even more. Even though
they are small in size, their density is close to solid matter and can thus
exhibit the same collective effects that are present in large scale
systems\cite{Reinhard2004}. Additionally, they are easily accessible
experimentally; simply forcing a gas through a small enough nozzle will
``stick'' atoms together and clusters will emerge from the nozzle's pin
hole. Varying the gas pressure will result in different cluster sizes
and they will normally have a structure in icosahedral
shells\cite{Reinhard2004}.
Different combinations of atoms are also possible, for example getting a core of
one kind of atom surrounded by shells of another element. More importantly for
the following work is the fact that clusters of atoms are small enough to be
studied numerically. Complexifying a system by adding more particles will
result in extra computational burden, so smaller systems are obviously easier
to simulate. Solving quantum mechanically an atom with many electrons is an
extremely hard problem. A system composed of dozens of atom is thus intractable
without approximations. Keeping the system as small as possible allows to
reduce to a maximum the number of approximations and stay as close as possible
to the physical system.

Another interesting aspect of clusters is the relative importance of
their boundaries compared to their volume. Because the surface of a sphere
scales as radius squared and the volume as radius cubed, smaller objects can
exhibit more boundary effects.

Experiments with clusters have shown interestingly large energy absorption
compared with gas\cite{Wabnitz2002,Bostedt2010}. High energetic electrons (tens
of keV) have been observed.
Ions can also be accelerated to MeV energies and X-rays can be generated by the
electrons' movement around ions in the cluster\cite{Ramunno2008}.

This brings important questions; do we fully understand how the energy is
absorbed by the clusters from the laser? How is that energy distributed
inside a nanoparticle? What are the different mechanisms that play a role in
the interaction? What is the effect and importance of these mechanisms? These
are the kind of questions theoretical studies can answer.

Certain heating mechanisms are known. Three of them are of great importance.
First, single- or multi-photon ionization is the absorption by a bound electron
of one or more photons which promotes this electron to the continuum. This
mechanism is dependent on the photon density; higher intensities mean more
photons which in turns increases the probability of an electron absorbing one
(or more) photon. The resulting electron will have, as its kinetic energy, the
remaining between the ionization potential energy and the photon energy. This
extra kinetic energy is clearly visible in electron spectra obtained from
experiments\citeneeded.

Single photon ionization is obviously dependent on the photon's wavelength; if
the photon's energy is not enough to bring the electron to the continuum,
ionization cannot take place. Multi-photon ionization requires intensity
regimes outside the present work requirement; it will thus not be described
here. A second heating  mechanism though is not
(directly) dependent on the wavelength but on the laser field strength. If the
laser pulse's intensity is strong enough it will bend the ion's Coulomb
potential. The electronic wavefunction will thus be able to tunnel
through the potential barrier and the probability of finding electrons
outside their parent atom (or ion), an ionization event, will increase. This
process if called \textit{tunnel} or \textit{field} ionization  since it
requires strong laser fields to happen. Laser intensity and atom type are the
main influence of tunnel ionization and the Keldysh parameter will give
information on whether the process is probable or not. Tunnel ionization is the
heart of the High Harmonic Generation (HHG) process; an electron tunnels out of
its parent ion potential and is then accelerated into the laser
field. When the field polarity is inverted during its oscillation cycle the
electron will be forced back to the parent ion where it recombines, emitting
high energy photons\citeneeded.

The two previous ionization processes (tunnel and single photon) are dominant
at the start of the laser pulse since they are the only way for the system to
absorb energy in the absence of free electrons. But once free electrons are
created by these processes a third one appears: impact ionization where an
energetic electron collides with an atom or ion and transfer some of its
kinetic energy to a bound electron, pushing it to the continuum.

FIXME: Talk about quasi-free vs free electrons?

After being created, free electrons can absorb energy from
the laser during the \textit{Inverse Bremsstrahlung Heating} (IBH) process.
Bremsstrahlung is the emission of photons during electron collisions with
heavier charged particles (ions). IBH is simply the opposite: electrons
absorbing photons in the presence of ions and being deflected. IBH is
an important heating mechanism at long wavelength (800 nm for example) while
being marginally efficient at shorter wavelength regimes (X-rays)\citeneeded.

The previous processes, while conceptually simple and elegant, are not
enough to explain high charge states seen in recent experiments at FEL
facilities. For example, Xenon's first two ionization potentials being 12.3~and
24.8~eV, they can be ionized by photons with wavelength shorter than 101~nm and
50~nm, respectively. Experiments with FEL pulses on Xenon clusters in 2002 saw
charge states up to Xe$^{8+}$ when using 100~nm photons\cite{Wabnitz2002}.
Many research groups have proposed new models to
explain the large absorption rates. The older model is called \textit{barrier
suppression} or \textit{barrier lowering} and is conceptually similar to tunnel
ionization. In this model, the potential barrier a bound electron sees will be
lowered by the presence of a neighbouring ion. This has been applied in
simulations to single photon ionization to allow less energetic photons to be
absorbed by electrons in a cluster environment\cite{Siedschlag2004}.

The second major model is based on \textit{atomic potentials} usage in
simulations instead of pure Coulombic ones. In all but hydrogen (or hydrogenic)
atoms, many electrons are present around the nucleus. Exact solutions (to the
non-relativistic \schrodinger equation) for many electron systems are not known
but are approximated by different methods. For example, the Hartree-Fock
method, used to solve for ground state electronic wavefunctions of a
multi-electron system, uses a mean field created by all other particles. This
mean field is used instead of a pure Coulomb one in simulations to portrait a
more realistic interaction between a moving electron and an
ion\cite{Greene2003}.

A third model has been proposed for describing large energy absorption by
clusters. This model is based on the density of electronic states just under
the continuum. There is an infinity of bound states; the lowest is the ground
state which is followed by excited states. As the level increases, the
difference in energy between them decreases. As the levels get closer and
closer together, distinguishing them energetically becomes harder and it is
possible to approximate them, in simulations, as continuum states instead. This
is the basis of the \textit{Many Body Recombination} (MBR) heating
process\cite{Jungreuthmayer2005}. Collisions between the high density electrons
let one loose some kinetic energy, becoming trapped in the ion's potential or
recombined to a highly excited state. Since the difference in energy between
this high excited state and the continuum is small, the electron could easily
absorb a new photon from the laser field and be promoted to the continuum where
the process can start over. This has the net effect of depositing an extra
photon worth of energy into the cluster environment.

The last model, introduced by our group, also explores the quantized nature of
the electronic states. Instead of looking at the highly excited states like
in the case of MBR the lowest energy levels are included as intermediate states
in the global energy transfer to the cluster. Normally during impact
ionization, an electron will need enough kinetic energy to bring a bound
electron from the ground state to the continuum as explained previously. This
put a limit on the number of electrons in the cluster environment that can
participate in such a process; all electrons with kinetic energy lower than the
ionization potential cannot impact ionize an atom or ion. The novel idea here
is to also include these electrons in the process by letting them transfer some
of their kinetic energy, through collisions, to a bound electron, bringing it
to an excited state. These excited states have a relatively long decay time (of
the order of nanosecond\citeneeded) and thus ``stores'' energy from free
electrons, letting them absorb even more energy from the laser field through
IBH or MBR. In the second step of the process, a collision from a second
electron can promote the first electron to the continuum. Because the
two impacting electrons don't need as much energy as in the case of impact
ionization alone, many more electrons can participate in this two-step
process. We called this process \textit{Augmented Collisional Ionization} (ACI).

It is clear that many different processes must be taken into account to study
the deposition of energy from a laser into nanoscale objects and its subsequent
redistribution. Theoretical investigations are essentials for better
understanding of the role of each processes.

