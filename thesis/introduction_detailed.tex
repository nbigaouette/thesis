\subsection{Detailed Introduction}
% Describe in more details the work done: The physics, the models. The specific
% litterature search goes here.

The previous section described the general goals of this thesis and was used to
give the reader an overview picture of the field. The following will go
into more details.

\subsubsection{Goals}
In view of the previous section, the goal of the current thesis is to increase
to global knowledge of laser-matter interaction. More specifically, the
question of \textit{how} the energy is deposited in a nanoscale object by an
ultra-short and ultra-intense laser pulse is important.

\cite{Young2010,Chapman2011}

\subsubsection{Laser-Cluster interaction}


\subsubsubsection{Clusters}
\label{section:intro:clusters:clusters}

Clusters of atoms are an extremely useful tool to study laser-matter
interaction. They are experimentally easy to produce since many
years\cite{Haberland1994}. Their size and constitution can be varied almost at
will \cite{Martin1996}.

\cite{Fennel2010}

\subsubsubsection{Heating mechanisms}
\label{section:intro:clusters:heating}

\subsubsubsection{Studied parameters}


\subsubsection{Methodology and Tools}
A large selection of tools exist to study laser-cluster interaction,
differentiating themselves through the amount of approximation taken.

Exact solution of the quantum mechanical system is, most of the time,
intractable. Theoretical investigations thus require some degree of
approximation, a compromise between feasibility and exactitude. On one end of
the spectrum, the most general methods is solving the
Time-Dependent \schrodinger Equation (TDSE) directly or the Quantum Monte-Carlo
(QMC) method. Unfortunately, these methods can only be applied to the simplest
systems of small numbers of electrons; clusters cannot be studied using these
methods.

Larger systems can be studied using \textit{ab initio} methods (``from
first principles'') which covers a wide range of techniques. In this class of
methods one can find \textit{Hartree-Fock} (HF) methods which consist on
approximating the ground state wavefunction by a single Slater determinant.
In HF methods, instantaneous electron-electron Coulomb repulsion is not
included directly in the system's Hamiltonian. Instead, only the average field
resulting from other electrons is used, giving the often used name of
\textit{self-consistent field} methods. Other \textit{ab
initio} methods are the \textit{Post Hartree-Fock} methods where electron
correlation is added. An example is the \textit{Configuration Interaction} (CI)
method. Because of their great accuracy, these methods are restricted to
relatively small systems, generally less then 10 atoms. Full \textit{ab initio}
treatment of clusters is not possible.

Larger systems requires more approximations. \textit{Density Functional Theory}
is an often used method for cluster studies requiring quantum aspects, with
either quantum or semiclassical propagation. It starts by formulating an
expression for the total energy of electrons and ions and derives static and
dynamic equations from it. All approximations are done in the selection of this
(energy) functional. The upper limit on these methods is of practical reasons,
mainly computational power available. On the other side, because the chosen
functional approximate the underlying quantum system, specific quantum effects
might not be included, for example shell effects or tunnelling are neglected.

Because DFT methods are mean field in nature, they cannot account for the large
field fluctuations seen in strong field cluster dynamics.

On the end of the methods spectrum lies the rate equations.






For a detailed review of the different methods, see Ref. \cite{Fennel2010}.



\subsubsubsection{Molecular Dynamics (MD)}
Due to the high charge states seen in experiment with clusters the only
practical method to microscopically study the ionization dynamics is
\textit{Molecular Dynamics} (MD) methods where ions and electrons are treated
classically.

In MD, particles interact directly through classical instantaneous forces. The
total force acting on particle $i$ of mass $m_i$ from all other $N$ particles in
the system is:
\begin{align}
m_i \va_i & = \vF_i = \sum_{j \ne i} \vF_{j \rightarrow i}
\label{eqn:md:newton}
\end{align}
In the present work, the force between charged particles is the instantaneous
electrostatic Coulomb force:
\begin{align}
\vF_{C,j \rightarrow i}\pa{\vr} & =\frac{k q_i q_j}{r_{ji}^2} \hvr_{ji}.
\label{eqn:md:coulomb:F}
\end{align}
which only depends on the distance between particles. Equation
\eqref{eqn:md:newton} can be time integrated using the Velocity-Verlet (VV)
scheme:
\begin{subequations}
\begin{align}
\vx_{i}^{\pa{n+1}} & = \vx_{i}^{\pa{n}} + \vv_{i}^{\pa{n}} \Delta t +
\frac{\va_{i}^{\pa{n}}}{2} \Delta t^2, \\
\va_{i}^{\pa{n+1}} & = \frac{\vF^{\pa{n}}}{m_i} \label{eqn:md:vv:a} \\
\vv_{i}^{\pa{n+1}} & = \vv_{i}^{\pa{n}} + \frac{\va_{i}^{\pa{n}} +
\va_{i}^{\pa{n+1}}}{2} \Delta t,
\end{align}
\label{eqn:md:vv}
\end{subequations}
where $\Delta t$ is the integration time step, $\vx_{i}$ the position vector,
$\vv_{i}$ the velocity vector and $\va_{i}$ the acceleration vector of
particle $i$, all evaluated at either the time step $n$ or the next one $n+1$.
Equations \eqref{eqn:md:vv}, when applied to every particles $i$ of the system,
can thus be used to propagate in time the whole cluster. Note that many
variation of equations \eqref{eqn:md:vv} are possible but all are equivalent.

Every particle in the system stores its position $\vx^{\pa{n}}$, its velocity
$\vv^{\pa{n}}$ and also the total force acting on it $\vF^{\pa{n}}$. This total
force is the sum of all contribution of equation \eqref{eqn:md:coulomb:F} from
all other particles in the system.

The MD algorithm basically calculates the force between every pair of particles
in the system. Since there is $N$ total particles, there is $O\pa{N^2}$
interactions to calculate. Doubling the number of particles will quadruple the
computational burden, effectively putting an upper limit on the number of
particles that can be simulated to tens of thousands.

An important problem to consider is the close range behaviour of equation
\eqref{eqn:md:coulomb:F} which diverges. Additionally, electrons should not be
able to classically recombine to an ion under the atomic energy level. To
prevent the later, electron recombination, as described in section
\ref{section:intro:clusters:heating}, can be enabled. But this does not prevent
the divergence of the Coulomb force. Instead, the problem is resolved by
changing the shape of equation \eqref{eqn:md:coulomb:F} at close range.

Different \textit{smoothing potentials} can be used to prevent the
discontinuity of the Coulomb potential (or force). An efficient way is to treat
electrons as charge distributions instead of point particles. As such, the
electrostatic potential due to a charge particle $j$ (of gaussian shape of
width $\sigma$) at location $\vr = r \hvr$ is:
\begin{align}
\phi_{j}\pa{\vr} & = \frac{k q_j}{r} \erf{\frac{r}{\sigma
\sqrt{2}}}
\end{align}
The associated electrostatic field is thus:
\begin{align}
\vE_{j}\pa{\vr} & = -\grad{\phi\pa{\vr}} = k q_j \pa{
    \frac{ \erf{\frac{r}{\sigma\sqrt{2}}} }{r^2}
    - \sqrt{\frac{2}{\pi}} \frac{ \ex{-\frac{r^2}{2 \sigma^2}} }{\sigma r}
}
\end{align}
When the distance $r$ is large compared to $\sigma$, the error function
is close to 1 and the potential becomes Coulombic. Also, the exponential
term in the electric field will tend towards 0 (since it's a gaussian shape).
The error function will tend towards 1, so the electric field will
be the field of a discrete point charge.







POTENTIAL SMOOTHING

RECOMBINATION

COMPUTATIONAL COST

\subsubsubsection{Quantum FDTD (QFDTD)}
\subsubsubsection{Acceleration through OpenCL and GP-GPUs}


\subsubsection{Thesis Outline}
Don't forget a review/description of each sections of the thesis.



\ReferencesSubsection{references_intro}
