\section*{Lora's comments}
\addcontentsline{toc}{section}{Lora's comments}

\subsection*{2013 05 14}
\addcontentsline{toc}{subsection}{2013 05 14}

\begin{itemize}
\item \sout{Cite: Varin2012, Peltz2012. Pre-ionized
clusters. Ultra large clusters: new frontier of modeling.}
\item One or two sentence(s)s at end of 1.1 about wavelength (???)
\item Add more details: context. More on size of clusters, scale. More details
people not familiar with clusters and/or lasers.
\item Add something about Ditmire's 3 step model explaining long wavelength stuff
(ionization, ?, coulomb explosion) --> Cite Ditmire (nanoplasma?)
\item In 1.1, add why laser-solid interaction is important --> LR will send ref.
on dielectrics.
\item We want a general picture for Carleton examiner who will know nothing about
clusters/lasers.
\item \sout{Add in section \ref{section:intro:mechanisms} a reference to
attosecond science and state that we are not interested in that.}
\item \sout{Top of page 2: add that all work is in femtosecond duration, leading to
attosecond.}
\item \sout{State that their is two kinds of absorption:}
    \begin{itemize}
    \item \sout{direct from laser field (photo-, tunnel, etc.)}
    \item \sout{indirect with help of environment (collisions, IBH, MBR, ACI...)}
    \end{itemize}
\item State that this thesis talks about ACI.
\item \sout{State that tunnel is at long wavelength, basis for attosecond}
\item \sout{Change ``heating'' to ``microscopic mechanisms underlying the dynamics''}
\item Change organization of heating mechanisms: start at long wavelengths, going
to shorter ones instead of stating each mechanisms.
\item \sout{Add a figure showing the different regimes (only for rare gas), intensity
vs wavelength and where each mechanisms is important. See Fennel2010 for a
similar figure. State where my contributions are.}
\item Know and state at which intensity multiphoton gets important; I'll get
question on it.
\item \sout{Put table of Ips for Argon and Xenon in 1.2}
\item \sout{Add something about the fact that every atoms has its own Ip and is thus
not influence the same way by the wavelength: then it's clear photon needs
enough energy to ionize.}
\item Add Bosted2009 ref. about frustration.
\item State that ``1.2.5 - Other mechanisms'' is in the VUV
\item Say something in 1.2 that it's ``well known'' (???)
\item In 1.2.5, add something like ``New experiment in VUV showed interesting
stuff, couldn't be explained. Had to introduce new mechanisms.''
\item \sout{In 1.2.6, add ``This thesis will show this''}
\item \sout{Put recombination paper in, before QFDTD.}
\item State that 1.2 is a motivation for the thesis
\item Merge the goals section into 1.2, since it's the motivation
\item Change title of ``1.4 - Thesis Outline'' to ``My contributions'' or
something like that.
\item Section 2.7 - Libraries: State that cross-sections were calculated/obtained
by Eddie.
\item Section 2.7 - Libraries: How are the cross-sections calculated? obtained?
Put this explanation into chapter 2 since it's technical.
\item Add more details on the ionization library in chapter 2 (own subsection).
Say something about the flags?
\item Expand more on the OpenCL implementation since I passes a lot of time on
it: show amount of work. Add a figure of how it works. Show a graph of speedups?
\item Say that QFDTD was used to test the Vb approximation
\item Change ``Conclusion'' to ``Discussion'', expand to ~10 pages.
\item Change ``Final words'' to ``Conclusion'', expand to ~2 pages.
\item \sout{Put reference section at the end.}
\item Add citations in the discussion.
\item Expand the discussion with more paragraphs.
\end{itemize}

Summarized:
\begin{enumerate}
\item Change the heating section. Re-organize by going from long wavelengths to
      shorter ones, with the mechanisms important at each regimes. Experiments
      in VUV saw something interesting, unexplained: required new models.
\item Expand Discussion
\item Expand Conclusion
\item Expand on cluster size influence. See Fennel2010\cite{Fennel2010} page 1796 (4)
      ``A. Basic cluster properties and time scales''.
\item Section 1.2 should be about the motivation of the thesis, not just plain
      description of mechanisms.
\item Expand ionization library section, put it in own real subsection. Where
      does cross-sections come? How are they calculated?
\item Add a Vb section --> goal of QFDTD is to test this.
\end{enumerate}


\newpage
\addcontentsline{toc}{subsection}{List of Corrections (FIXMEs)}
\listoffixmes
