\subsection{Libraries}
\label{section:tools:libraries}

Additionally to the MD and QFDTD packages, eight libraries were developed to
support some generic features shared between the two. These libraries are:
\begin{itemize}
\item timing.git\footnote{ \url{
    https://gitlab.cphoton.science.uottawa.ca/nbigaouette/timing}}:
    Timing routines used to measure running time, estimated time of arrival
    (ETA), code profiling, etc. 1,300 lines of C++ code.
\item stdcout.git\footnote{ \url{
    https://gitlab.cphoton.science.uottawa.ca/nbigaouette/stdcout}}:
    Logging features allowing saving the output of any simulations to a
    (compressed) log file. 500 lines of C++ code.
\item prng.git\footnote{ \url{
    https://gitlab.cphoton.science.uottawa.ca/nbigaouette/prng}}:
    Pseudo-random number generator (PRNG) library. Wrapper around
    dSFMT\cite{prng2009}, a SIMD-oriented Fast Mersenne Twister implementation.
    Defines easy to use PRNG functions, distributions, seed, etc. Keeps track of
    the seed and how many times pseudo-random numbers were generated, allowing
    to replay the series in case of simulation reloading. 500 lines of C++ code
    (not counting dSFMT).
\item memory.git\footnote{ \url{
    https://gitlab.cphoton.science.uottawa.ca/nbigaouette/memory}}:
    Wrappers around malloc() and calloc() that will automatically check that
    the allocation succeeded. Will also keep track of the amount of
    memory allocated and allow setting a maximum value, preventing
    over-allocation due to bugs or user errors. Can also return the binary
    representation of a number as a string for easier debugging. 1,200 lines of
    C++ code.
\item io.git\footnote{ \url{
    https://gitlab.cphoton.science.uottawa.ca/nbigaouette/io}}:
    Input and Output library. Includes wrappers around TinyXML library\cite{tinyxml} for
    reading input file, wrappers around NetCDF\cite{netcdf} for self-contained
    output files, used for both post-processing and simulation snapshots.
    2,950 lines of C++ code (not counting TinyXML).
\item libpotentials.git\footnote{ \url{
    https://gitlab.cphoton.science.uottawa.ca/nbigaouette/libpotentials}}:
    Functions implementing and abstracting different potential shapes as
    described in section \ref{section:intro:md:potentials}. 4,300 lines of C++ code.
\item ionization.git\footnote{ \url{
    https://gitlab.cphoton.science.uottawa.ca/nbigaouette/ionization}}:
    All ionization routines described in section
    \ref{section:intro:mechanisms}. 18,000 lines of C++ code.
\item oclutils.git\footnote{ \url{
    https://gitlab.cphoton.science.uottawa.ca/nbigaouette/oclutils}}:
    Library to ease the use of OpenCL devices. Allows listing and selecting the
    best GPU available on a workstation and locking it to prevent
    other simulations from using it. Contains an array abstraction to ease the
    transfer of data from the host (main memory) to device (GPU memory) and
    vice-versa. Also includes a SHA512 checksum check to validate memory
    and detect any issue during transfers. 3,300 lines of C++ code.
\item get\_libraries.git\footnote{ \url{
    https://gitlab.cphoton.science.uottawa.ca/nbigaouette/get_libraries}}:
    Simple script that will download all the previous required libraries or
    update to the lasted version from git, compile them and install them in the
    user's directory. 240 lines of bash code.
\end{itemize}
% A detailed guide on the compilation and usage of the simulation package can be
% read in appendix \ref{appendix:code}.

