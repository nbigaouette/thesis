\subsubsection{Libraries}

Additionally, a total of seven other libraries were developed to support some
generic features of the code. These libraries are:
\begin{itemize}
\item ionization.git\footnote{ \url{
    https://gitlab.cphoton.science.uottawa.ca/nbigaouette/ionization}}:
    All ionization routines described in section
    \ref{section:intro:clusters:heating}
\item timing.git\footnote{ \url{
    https://gitlab.cphoton.science.uottawa.ca/nbigaouette/timing}}:
    Timing routines used to measure running time, estimated time of arrival
(ETA), etc.
\item stdcout.git\footnote{ \url{
    https://gitlab.cphoton.science.uottawa.ca/nbigaouette/stdcout}}:
    Logging features allowing saving the output of any simulations to a log
    (compressed) file.
\item prng.git\footnote{ \url{
    https://gitlab.cphoton.science.uottawa.ca/nbigaouette/prng}}:
    Pseudo-random number generator (PRNG) library. Wrapper around
dSFMT\cite{prng2009}. Defines easy to use PRNG functions, distributions, seed,
etc. Keeps track of how many times it has been called for easy reloading
exactly where simulation left off.
\item memory.git\footnote{ \url{
    https://gitlab.cphoton.science.uottawa.ca/nbigaouette/memory}}:
    Wrappers around malloc() and calloc() that will automatically check that
    the allocation succeeded. Will also keep track of the amount of
    memory allocated and allow setting a maximum value, preventing
    over-allocation due to bugs or user errors. Can also print values directly
    in binary for easy debugging.
\item io.git\footnote{ \url{
    https://gitlab.cphoton.science.uottawa.ca/nbigaouette/io}}:
    Input and Output library. Wrappers around TinyXML library\cite{tinyxml} for
reading input file. Wrappers around NetCDF\cite{netcdf} for self-contained
output files, used in simulation snapshots.
\item libpotentials.git\footnote{ \url{
    https://gitlab.cphoton.science.uottawa.ca/nbigaouette/libpotentials}}:
    Functions implementing and abstracting different potential shapes.
\item oclutils.git\footnote{ \url{
    https://gitlab.cphoton.science.uottawa.ca/nbigaouette/oclutils}}:
    Library to ease the use of OpenCL devices. Allows selecting the best GPUs
    available on a workstation and locking it to prevent other simulations from
    using it.
\item get\_libraries.git\footnote{ \url{
    https://gitlab.cphoton.science.uottawa.ca/nbigaouette/get_libraries}}:
    Simple script that will download all the previous required libraries,
    compile them and install them in the user's directory.
\end{itemize}
A detailed guide on the compilation and usage of the simulation package can be
read in appendix \ref{appendix:code}.

