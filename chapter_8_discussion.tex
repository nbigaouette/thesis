\section{Discussion}

Clusters of atoms are without a doubt an extremely useful tool to study
laser and matter interaction. They were in use for their optical properties
even before atoms were discovered. With the advancement of both cluster creation
and laser technologies, it became possible to gain a significant understanding
on how light interacts with matter. Cluster studies then became an important
physics research area.

At the simplest level, quantum mechanics describes the dynamics but exact solutions
are not possible for systems of more than two electrons.
The process of photo-ionization of single atoms is known, but the high density
of atoms inside clusters gives rise to interesting phenomena. It was a surprise
to scientists ten years ago to see highly charged ions coming out of experiments
using rare gas clusters with short duration (tens of femtoseconds), high intensity
(10$^{12}$ -- 10$^{14}$ W/cm$^2$) and small wavelength (VUV, XUV) laser pulses.
More than ten years later, our understanding of the different mechanisms of energy
absorption and diffusion throughout the cluster is taking shape. Even though full
quantum mechanical description of clusters is not possible, some level of approximations
can be taken, allowing theoretical and numerical studies to be performed.

Our numerical tool of choice is Molecular Dynamics where we track, classically,
the dynamics of electrons and ions during and after a laser pulse. MD simulations,
being classical systems, allows a larger number of particles than full quantum
simulations but can still describe the large fluctuations in charge and field
generated inside a cluster.

Many MD packages are available freely that could have been used for
the current work. After a first try with one such package, it was decided to
start from scratch a new one. The MD algorithm is relatively simple; having
control on the rest of the package allowed flexibility in the development that
would have not been possible with a third party package. For example, existing
MD codes have specific target users that do not have the same requirements as what
was required for this work, such as varying the number of particles during the
simulation or long range interactions. By building the tools myself I was able
to have more trust in it by subjecting it to rigorous tests and validations.
Additionally, even though a lot of work was put into the creation of this code
base, much time was saved trying to understand, or even worst, debug, a code from
someone else. Having known at first the amount of time lost on an unstable code
base, I would have started my MD package right away, as it proved a lot easier
to move forward with it.


The $O(N^2)$ scaling of the MD algorithm prevents simulations of more than tens
of thousands of particles but interesting results are still accessible.
For bigger systems, some algorithm changes can be done to speed the force
calculation on particles. Using a hierarchical tree can change the algorithm
scaling from $O(N^2)$ to $O(N \textrm{log}\pa{N})$ and thus push the limit on
cluster size upward. Having control and full understanding of the code had
the added benefit of easing the process of adding new feature. While the tree
algorithm is implemented, it was not yet used in a publication. An interesting
technology matured during the last decade; by running codes directly on GP-GPUs,
the same algorithms could perform up to one hundred times faster than on
conventional CPUs. The MD package was thus ported to the framework OpenCL where
the MD algorithm showed a speedup of close to 80. This implementation was used
in our publications since the cluster sizes that we were interested in were
in the range accessible through MD (and not requiring the tree). The
hierarchical tree-based algorithm is still accessible through an input parameter
(not as an OpenCL implementation though), making my MD package interesting for
future studies where larger clusters will be investigated.

% ******************************************************************************
In our first publication in 2011, the new model using excited states as a
multi-step of energy transfer process from a laser pulse to rare gas clusters
was introduced. This model is based on electron collisions and increases the
amount of ionization in clusters. We thus dubbed this model Augmented
Collisional Ionization, or ACI. With ACI, many more electrons can participate in
collisional ionization.

The first study concentrated on argon clusters in the 32.8 nm (37.8 eV) regime.
At this wavelength, only the first two ionization states of (atomic) argon are
accessible but experiments in 2008 saw up to Ar$^{4+}$. The small cluster sizes
used in the experiment ($\sim$100 atoms per cluster) allowed to quickly test and
validate our model. It was found that ACI plays a crucial role in the cluster
dynamics and can explain the high charge states seen in the experiment, with
intensities from $\ten{5}{13}$ to $\ten{10}{13}$~W/cm$^{2}$. When ACI was
enabled in the simulations, the Ar$^{4+}$ population closely matched what was
seen at FLASH. Additionally, the MD simulations allowed tracking the dynamics
of each charge states. It was found that all Ar$^{3+}$ and Ar$^{4+}$ were
created by ACI, making it a vital process in cluster dynamics where traditional
ionization mechanisms cannot explain alone what is seen in laboratories.

% ******************************************************************************
In the second publication, the size of argon clusters was investigated. We found
that as clusters get bigger and bigger -- from Ar$_{55}$ to Ar$_{2,057}$ --
the importance of collisions increased. As clusters ionize progressively they
become transparent to photons; ACI accelerates the rate at which Ar$^{2+}$ are
created and Ar$^{2+}$ cannot be photo-ionized (37.8 eV photon vs. 48.7 eV IP).
This process was baptized collisionally reduced photoabsorption. Once clusters
are ionized, the bigger ones will thermalize faster and the electron velocity
distribution will become isotropic in contrast, with smaller clusters which
keep their photo-ionization-created anisotropy.


% ******************************************************************************
Later on, a question presented itself; What is the influence of the cluster
environment on the different states? One of the first books on the
electrodynamics solver called FDTD by Sullivan contained a section on using
this method to solve the \schrodinger equation. Since I was familiar with
the electrodynamics FDTD that I implemented during my masters degree I decided
to explore the quantum version of it. Even though the current implementation is
a single electron picture, a study on the cluster states could still be possible,
at least as a first step. A major problem with FDTD is the amount of memory
required to resolve a three dimensional grid. Normally, FDTD is parallelized
using a shared memory model so it can be distributed on many
computational nodes. After the work done on implementing the MD algorithm in
OpenCL to run on a GP-GPU, I decided to port the QFDTD to OpenCL too
which was actually relatively quick. The video cards could run the code faster,
but video cards have their own discrete memory which is a lot smaller than main
memory. A novel mapping method was thus developed to reduce the number of grid
cell size in the QFDTD computational domain. This new method's innovation comes
from the two space being mapped. Traditionally, one space is mapped to another
one where some physical feature is either enhanced or reduced to ease the numerical
burden. In my nonlinear mapping, one of the space mapped is the actual
discrete memory of the computer; the physical space is mapped to the
integer-referenced memory location through a cumulative sum of a source function,
enforcing the monotonicity required. Due to the way the mapping function is
obtained, any source function can be used, giving great flexibility to the
mapping process. It is thus possible to concentrate grid points around centres
of interest, in this case ions' location. In addition to the nonlinear mapping,
both real-time and imaginary-time methods were implemented, allowing to compare
the two through the same package. In the case of the imaginary method, a new
way to obtain the eigenstates was described and successively used on the
hydrogen atom and the simplest molecule: hydrogen cation molecule (H$_{2}^{+}$).
Using the real-time method, it was possible to see ten excited states. This
publication allowed to validate the whole technique, including the nonlinear
mapping, and compare with known values from experiments.

% ******************************************************************************
The ACI model, based on electron impact excitation cross-sections, is independent of the
regime and does not depend on the laser parameters (pulse duration, wavelength,
intensity, etc.). However, transition cross-sections are unique to each element
and must be calculated and added to the simulation package before use. Argon
and xenon cross-sections where generated and are accessible. It is thus
interesting to apply the model to different wavelength regimes. Other models were developed
to explain Wabnitz \textit{et~al.}'s 2002 VUV-xenon experiment at FLASH but when
Bostedt \textit{et~al.} published in 2010 a revised intensity (to lower values) it was
not clear if these models could still be able to reproduce the experimental results
in light of the new, lower intensity. We decided to revisit the 2002 100~nm
experiment in light of our ACI model at the revised, lower intensity.

Thanks to my OpenCL implementation (and its 80 times speedup), it was possible
to acquire significant statistics over a wide range of intensities, effectively
reproducing the cluster distribution in the laser focus spacial profile. The
speedup gained was used to increase the number of simulations, thus increasing
the statistical significance of the data, and to acquire data at different
intensities to sample a non-uniform spacial distribution of the laser pulse.

We showed that ACI increased the maximum charge states observed by two states in
Xe$_{80}$ clusters at $8\times10^{12}$~W/cm$^2$ and Xe$_{1000}$ clusters
at $1.5\times10^{13}$~W/cm$^2$. This maximum charge state went from Xe$^{3+}$ to
Xe$^{5+}$ for the smaller clusters and from Xe$^{5+}$ to Xe$^{7+}$ for the
larger clusters. For the simulation results to match the 2002 DESY experiment,
ACI had to be enabled, a clear indication that ACI plays an important role in
the cluster dynamics.

We then looked at the cluster size influence on the charge state spectra
at \\ $\ten{8}{12}$~W/cm$^2$. For the shapes to match the
experimental data, a spacial averaging of the intensity in the laser profile
had to be performed. ACI was vital to obtain both the same maximum charge state
(Xe$^{4+}$) and the same most abundant one (Xe$^{1+}$) for Xe$_{90}$ clusters
as the experiment.

The influence of the potential depth used in the simulations was also
investigated. By increasing IBH, a deeper potential allowed an increase of the
maximum charge state seen as well as the most abundant one, up to a depth
of around 27.2~eV (1~Eh), when ACI was enabled.


\subsection{Future direction}

The largest clusters simulated were Xe$_{5,083}$ (11
icosahedral shells), much smaller than the experiment's Xe$_{90,000}$
($\sim$30 icosahedral shells), a cluster size not possible with traditional
MD simulations. The MD's $O\pa{N^2}$ scaling makes it extremely difficult
to increase the number of particles past a couple of thousands. While
the OpenCL implementation is a nice addition that gives almost two orders
of magnitude speedup, the algorithm's scaling is always present.
Parallelizing the MD algorithm is possible but a difficult task which does
not free one from the its scaling. To push cluster sizes a different
algorithm will have to be used.

The tree algorithm would be the first step due to its $O\pa{N \log N}$ scaling.
The algorithm is already implemented but would require more testing
before being used in production. Its advantage would be to use the
ionization library developed in the present work.

Another approach could be taken instead,
mainly switching from a direct particle-to-particle (PP) algorithm to a more
scalable particle-mesh (PM) or even particle-particle, particle-mesh (P3M)
technique. PIC for example scales as $O\pa{N_c^3}$ for the number of grid cells and
$O\pa{N_p}$ for the number of particles.

Simulating larger clusters is essential to reach the scale of biomolecules.
While experiments have already been performed on some, for example
on the mimivirus or photosynthesis proteins, theoretical tools are still
in their infancy.

The effects of excited states and ACI in clusters was shown to be significant
in different intensity and wavelength regimes. Further investigation might
reveal ACI to be present in other type of molecules or at other regimes.

The Quantum FDTD model showed interesting potential for further investigation.
While still based on the one electron picture, the potential used can
be arbitrary in three dimensions. By changing the Coulombic potential
for a Hartree–Fock self-consistent field, a multi-electrons system
could be approached. Using this atomic potential could lead to a comparison
with reference~\cite{Walters2006}.
Sullivan did develop a two-electron QFDTD using the
Hartree-Fock formulation but his systems were mostly 2D, considerably reducing
the computational burden required. My package could be used to simulate one
electron in an effective potential, using the nonlinear mapping to make the
calculation possible.

Additionally, the potential used in the (real time) QFDTD does not have to be
time independent. Using a time dependent potential could simulate the effect of
a (classical) laser field (through the vector potential). It is hypothesized
that transition rates could be obtained from the output of the real time QFDTD.
If rates and cross-sections could be obtained for an isolated atom, the
influence of the neighbouring ions and the cluster environment could be
calculated.

Going forward, it will
be possible to validate and study some approximations used in the MD package,
for example, what is the influence of the shape of the cluster environment,
approximated as a constant $U_b$, over the cluster dynamics? How much is
an electron wavefunction affected by a neighbouring ion?

Many tools were developed for the present work. These tools still show
great promises for future work.



\subsection{Final words}

Theoretical laser-clusters studies is a nice amalgam of many challenging
fields of physics. Clusters, being nanoscopic objects, live in a quantum world
where particles are described by wavefunctions. When free inside the cluster
the particles' dynamics can be treated classically through Newtonian mechanics.
Unfortunately, even three hundred years after Newton, the many-body problem,
also showing up in astronomy, is not solvable analytically.
Computers must be used to calculate these simple trajectories
using many algorithms, bringing computer science into the picture
with all its quirks, problems, bugs, crashes and high expectation algorithms or
new technologies. Lasers are the tool of choice to study the small objects
clusters. Electrodynamics, classical or quantum, describes light, the purest
form of energy. After clusters get ionized,
the plasma generated can be described with the language of statistical mechanics.
Additionally to all these fields, clusters can be used as a first-order model
for biomolecules or can be used in biological systems. As if physics wasn't
complex enough, life and biology bring so many more questions to the picture.

It is easy to be impressed and even overwhelmed by the wide range of fields
touched by clusters studies, but how satisfying can it be to reach so many
topics at once? Clusters are quite fascinating and I hope to have
brought my little contributions, through this thesis, to the height of the giants'
shoulders so next generations will see even further.
