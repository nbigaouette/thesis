%%%%%%%%%%%%%%%%%%%%%%%%%%%%%%%%%%%%%%%%%%%%%%%%%%%%%%%%%%%%%%%%%%%%%%%%%%%%%%%%%%%%%%
%%%%%%%%%%%%%%%%%%%%%%%%%%%%%%%%%%%%%%%%%%%%%%%%%%%%%%%%%%%%%%%%%%%%%%%%%%%%%%%%%%%%%%
%%                             Usefull macros                                       %%
%%                           Nicolas Bigaouette                                     %%
%%                          nbigaouette@gmail.com                                   %%
%%%%%%%%%%%%%%%%%%%%%%%%%%%%%%%%%%%%%%%%%%%%%%%%%%%%%%%%%%%%%%%%%%%%%%%%%%%%%%%%%%%%%%
%%%%%%%%%%%%%%%%%%%%%%%%%%%%%%%%%%%%%%%%%%%%%%%%%%%%%%%%%%%%%%%%%%%%%%%%%%%%%%%%%%%%%%

% Needed for \onehalf
% WARNING: This package breaks flushright!!!
% \usepackage{ltugcomn}
% Needed for bold Greek letters
\usepackage{bm}

\newcommand{\mailto}[1]{\href{mailto:#1}{#1}}

\newcommand{\angstrom}{{\AA}ngstr\"om}

% \newcommand{\vecnabla}{\vec{\nabla}}
\newcommand{\vecnabla}{\bm{\nabla}}
\newcommand{\laplacien}[1]{\vecnabla^2 #1}           % Laplacien
\newcommand{\laplacian}[1]{\laplacien{#1}}           % Laplacien
\newcommand{\laplacient}[1]{\vecnabla_{\bot}^{2} #1} % Laplacien transverse
\newcommand{\gradient}[1]{\vecnabla #1}              % Gradient
\newcommand{\grad}[1]{\vecnabla #1}                  % Gradient
\newcommand{\divergence}[1]{\vecnabla \cdot #1}      % Divergence
\newcommand{\rotationnel}[1]{\vecnabla \times #1}    % Rotationnel

% \newcommand{\spherical}{\ensuremath{\mathcal{Y}}_{l,m}\pa{\theta,\phi}}
%                                                     % Spherical Harmonics Y_lm
\newcommand{\spherical}{Y_{l,m}\pa{\theta,\phi}}         % Spherical Harmonics Y_lm
\newcommand{\sphericalp}{Y_{l,m}\pa{\theta',\phi'}}      % Spherical Harmonics Y_lm
\newcommand{\sphericallmp}{Y_{l',m'}\pa{\theta,\phi}}    % Spherical Harmonics Y_lm
\newcommand{\sphericalpc}{Y_{l,m}^*\pa{\theta',\phi'}}   % Spherical Harmonics Y_lm
\newcommand{\sphericalc}{Y_{l,m}^*\pa{\theta,\phi}}      % Spherical Harmonics Y_lm
\newcommand{\gspherical}{\ensuremath{\mathcal{Y}}_{j,l,m}\pa{\theta,\phi}}
                                                         % Genralized Spherical Harmonics
                                                         % Y_jml

\newcommand{\onehalf}{\sfrac{1}{2}}
\newcommand{\onethird}{\sfrac{1}{3}}
\newcommand{\onefourth}{\sfrac{1}{4}}
\newcommand{\twothird}{\sfrac{2}{3}}
\newcommand{\threehalf}{\sfrac{3}{2}}
\newcommand{\fivehalf}{\sfrac{5}{2}}
\newcommand{\eighthalf}{\sfrac{8}{2}}
\newcommand{\summation}[2]{\sum\limits_{#1}^{#2}}

\newcommand{\im}{\textrm{i}}                                    % i = sqrt{-1}

\newcommand{\mum}{\mu \textrm{m}}                               % mu m

\newcommand{\hbartwo}{\frac{\hbar}{2}}
\newcommand{\twohbar}{\frac{2}{\hbar}}

% See:
% http://tex.stackexchange.com/questions/18298/prevent-line-breaking-in-inline-mat
% h-but-allow-flexible-spacing
\newcommand{\preventlinebreak}[1]{{}$\kern-2\mathsurround${}
  \binoppenalty10000 \relpenalty10000 #1{}$\kern-2\mathsurround${}}
\newcommand{\ten}[2]{\preventlinebreak{#1\times10^{#2}}}

\newcommand{\dd}[1]{~\textrm{d} #1}

\newcommand{\del}{\partial}                                     % del

\newcommand{\delexp}[3]{\frac{\del^{#3} #1}{\del #2^{#3}}}          % del^? ? / del?
\newcommand{\dexp}[3]{\frac{\textrm{d}^{#3} #1}{\textrm{d} #2^{#3}}}% d^? ? / d?

\newcommand{\delx}[1]{\delexp{#1}{x}{{}}}                     % del ? / delx
\newcommand{\dely}[1]{\delexp{#1}{y}{{}}}                     % del ? / dely
\newcommand{\delz}[1]{\delexp{#1}{z}{{}}}                     % del ? / delz
\newcommand{\delr}[1]{\delexp{#1}{r}{{}}}                     % del ? / delr
\newcommand{\delt}[1]{\delexp{#1}{t}{{}}}                     % del ? / delt
\newcommand{\deli}[2]{\delexp{#1}{{#2}}{{}}}                  % del ? / del?

\newcommand{\delxs}[1]{\delexp{{#1}}{x}{2}}                     % del^2 ? / delx^2
\newcommand{\delys}[1]{\delexp{{#1}}{y}{2}}                     % del^2 ? / dely^2
\newcommand{\delzs}[1]{\delexp{{#1}}{z}{2}}                     % del^2 ? / delz^2
\newcommand{\delts}[1]{\delexp{{#1}}{t}{2}}                     % del^2 ? / delt^2
\newcommand{\delrs}[1]{\delexp{{#1}}{r}{2}}                     % del^2 ? / delr^2
\newcommand{\delis}[2]{\delexp{{#1}}{#2}{2}}                    % del^2 ? / del?^2

\newcommand{\delxt}[1]{\delexp{{#1}}{x}{3}}                     % del^3 ? / delx^3
\newcommand{\delyt}[1]{\delexp{{#1}}{y}{3}}                     % del^3 ? / dely^3
\newcommand{\delzt}[1]{\delexp{{#1}}{z}{3}}                     % del^3 ? / delz^3
\newcommand{\deltt}[1]{\delexp{{#1}}{t}{3}}                     % del^3 ? / delt^3
\newcommand{\delrt}[1]{\delexp{{#1}}{r}{3}}                     % del^3 ? / delr^3
\newcommand{\delit}[2]{\delexp{{#1}}{#2}{3}}                    % del^3 ? / del?^3

\newcommand{\delxf}[1]{\delexp{{#1}}{x}{4}}                     % del^4 ? / delx^4
\newcommand{\delyf}[1]{\delexp{{#1}}{y}{4}}                     % del^4 ? / dely^4
\newcommand{\delzf}[1]{\delexp{{#1}}{z}{4}}                     % del^4 ? / delz^4
\newcommand{\deltf}[1]{\delexp{{#1}}{t}{4}}                     % del^4 ? / delt^4
\newcommand{\delrf}[1]{\delexp{{#1}}{r}{4}}                     % del^4 ? / delr^4
\newcommand{\delif}[2]{\delexp{{#1}}{#2}{4}}                    % del^4 ? / del?^4

\newcommand{\dx}[1]{\dexp{{#1}}{x}{{}}}                       % d ? / dx
\newcommand{\dy}[1]{\dexp{{#1}}{y}{{}}}                       % d ? / dy
\newcommand{\dz}[1]{\dexp{{#1}}{z}{{}}}                       % d ? / dz
\newcommand{\dt}[1]{\dexp{{#1}}{t}{{}}}                       % d ? / dt
\newcommand{\dr}[1]{\dexp{{#1}}{r}{{}}}                       % d ? / dr
\newcommand{\di}[2]{\dexp{{#1}}{#2}{{}}}                      % d ? / d?

\newcommand{\dxs}[1]{\dexp{{#1}}{x}{{2}}}                     % d^2 ? / dx^2
\newcommand{\dys}[1]{\dexp{{#1}}{y}{{2}}}                     % d^2 ? / dy^2
\newcommand{\dzs}[1]{\dexp{{#1}}{z}{{2}}}                     % d^2 ? / dz^2
\newcommand{\drs}[1]{\dexp{{#1}}{t}{{2}}}                     % d^2 ? / dr^2
\newcommand{\dts}[1]{\dexp{{#1}}{r}{{2}}}                     % d^2 ? / dt^2
\newcommand{\dis}[2]{\dexp{{#1}}{#2}{{2}}}                    % d^2 ? / d?^2

\newcommand{\dtreal}[1]{\frac{\textrm{d} #1}{\textrm{d} t}}         % d ? / dt
\newcommand{\dtsreal}[1]{\frac{\textrm{d}^2 #1}{\textrm{d}t^2}}     % d^2 ? / dt^2
% \newcommand{\dt}[1]{\dot{#1}}                                   % d ? / dt
% \newcommand{\dts}[1]{\ddot{#1}}                                 % d^2 ? / dt^2

\newcommand{\erf}[1]{~\textrm{erf} \left\{#1\right\}}                      % erf{?}
\newcommand{\ex}[1]{\exp{ \left\{ #1 \right\}}}                 % exp{?}
\newcommand{\ep}[1]{\textrm{e}^{\left( #1 \right) }}            % e^(?)
\newcommand{\e}[1]{\textrm{e}^{#1}}                             % e^?
\newcommand{\lnp}[1]{\textrm{ln} \left( #1 \right)}            % ln(?)
\newcommand{\logp}[1]{\textrm{log} \left( #1 \right)}          % log(?)

\newcommand{\cosinus}[1]{\cos #1 }                              % cos
\newcommand{\cosine}[1]{\cosinus{#1}}                           % cos
\newcommand{\sinus}[1]{\sin #1 }                                % sin
\newcommand{\sine}[1]{\sinus{#1}}                               % sin
\newcommand{\cosp}[1]{\cos \left( #1 \right)}                   % cos()
\newcommand{\sinp}[1]{\sin \left( #1 \right)}                   % sin()
\newcommand{\tanp}[1]{\tan \left( #1 \right)}                   % tan()
\newcommand{\sintheta}{\sin \theta}                             % sin theta
\newcommand{\costheta}{\cos \theta}                             % cos theta
\newcommand{\sinptheta}{\sinp{ \theta }}                        % sin(theta)
\newcommand{\cosptheta}{\cosp{ \theta}}                         % cos(theta)
\newcommand{\sinec}[1]{\textrm{ sinc} \left( #1 \right)}        % sinc()
\newcommand{\sinecsquared}[1]{\textrm{ sinc}^2 \left( #1 \right)} % sinc^2()
\newcommand{\cossquared}[1]{\cos^2 \left( #1 \right)}           % cos^2()
\newcommand{\sinsquared}[1]{\sin^2 \left( #1 \right)}           % sin^2()
\newcommand{\tansquared}[1]{\tan^2 \left( #1 \right)}           % tan^2()
\newcommand{\acos}[1]{\textrm{acos} \left( #1 \right)}          % acos()
\newcommand{\asin}[1]{\textrm{asin} \left( #1 \right)}          % asin()
\newcommand{\atan}[1]{\textrm{atan} \left( #1 \right)}          % atan()
\newcommand{\sech}[1]{\textrm{sech} \left( #1 \right)}          % sech()
\newcommand{\tanhyp}[1]{\textrm{tanh} \left( #1 \right)}          % tanh()
\newcommand{\sechsquared}[1]{\textrm{sech}^2 \left( #1 \right)} % sech^2()

\newcommand{\thetatwo}{\frac{\theta}{2}}
\newcommand{\betatwo}{\beta/2}

\newcommand{\dix}[1]{\times 10^{#1}}                            % ? x 10^?

\newcommand{\dint}[1]{~\textrm{d}#1}

\newcommand{\eps}{\epsilon}
\newcommand{\epsz}{\epsilon_0}
\newcommand{\epsr}{\epsilon_r}
\newcommand{\muz}{\mu_0}
\newcommand{\om}{\omega}
\newcommand{\omi}[1]{\omega_{#1}}
\newcommand{\omegaz}{\omega_0}
\newcommand{\omegasquared}{\omega^2}
\newcommand{\omegazsquared}{\omega_0^2}
\newcommand{\omegazsquaredomegasquared}{\omegazsquared - \omegasquared}

% \newcommand{\ve}[1]{\vec{#1}}
\newcommand{\ve}[1]{\textbf{#1}}
\newcommand{\vE}{\ve{E}}
\newcommand{\vA}{\ve{A}}
\newcommand{\vB}{\ve{B}}
\newcommand{\vC}{\ve{C}}
\newcommand{\vD}{\ve{D}}
\newcommand{\vF}{\ve{F}}
\newcommand{\vH}{\ve{H}}
\newcommand{\vJ}{\ve{J}}
\newcommand{\vL}{\ve{L}}
\newcommand{\vM}{\ve{M}}
\newcommand{\vP}{\ve{P}}
\newcommand{\vR}{\ve{R}}
\newcommand{\vS}{\ve{S}}
\newcommand{\vU}{\ve{U}}
\newcommand{\va}{\ve{a}}
\newcommand{\vb}{\ve{b}}
\newcommand{\vc}{\ve{c}}
\newcommand{\vd}{\ve{d}}
\renewcommand{\vee}{\ve{e}}
\newcommand{\vf}{\ve{f}}
\newcommand{\vg}{\ve{g}}
\newcommand{\vh}{\ve{h}}
\newcommand{\vi}{\ve{i}}
\newcommand{\vj}{\ve{j}}
\newcommand{\vk}{\ve{k}}
\newcommand{\vl}{\ve{l}}
\newcommand{\vm}{\ve{m}}
\newcommand{\vn}{\ve{n}}
\newcommand{\vp}{\ve{p}}
\newcommand{\vq}{\ve{q}}
\newcommand{\vr}{\ve{r}}
\newcommand{\vqd}{\dot{\vq}}
\newcommand{\vv}{\ve{v}}
\newcommand{\vs}{\ve{s}}
\newcommand{\vt}{\ve{t}}
\newcommand{\vx}{\ve{x}}
\newcommand{\vz}{\ve{z}}

\newcommand{\vomega}{\bm{\omega}}
\newcommand{\vmu}{\bm{\mu}}
\newcommand{\vepsilon}{\bm{\epsilon}}
\newcommand{\vDelta}{\bm{\Delta}}
\newcommand{\vxi}{\bm{\xi}}
\newcommand{\vsigma}{\bm{\sigma}}

\newcommand{\operator}[1]{ \hat{#1} }
\newcommand{\operatorvector}[1]{ \hat{\textbf{#1}} }

\newcommand{\oa}{\operator{a}}
\newcommand{\ob}{\operator{b}}
\newcommand{\oc}{\operator{c}}
\newcommand{\oi}{\operator{i}}
\newcommand{\oj}{\operator{j}}
\newcommand{\ok}{\operator{k}}
\newcommand{\op}{\operator{p}}
\newcommand{\os}{\operator{s}}
\newcommand{\ox}{\operator{x}}
\newcommand{\oy}{\operator{y}}
\newcommand{\oz}{\operator{z}}
\newcommand{\oA}{\operator{A}}
\newcommand{\oB}{\operator{B}}
\newcommand{\oD}{\operator{D}}
\newcommand{\oH}{\operator{H}}
\newcommand{\oJ}{\operator{J}}
\newcommand{\oK}{\operator{K}}
\newcommand{\oL}{\operator{L}}
\newcommand{\oO}{\operator{O}}
\newcommand{\oP}{\operator{P}}
\newcommand{\oS}{\operator{S}}
\newcommand{\oT}{\operator{T}}
\newcommand{\oU}{\operator{U}}
\newcommand{\oxi}{\operator{\xi}}


\newcommand{\ograd}{ \hat{\bm{\nabla}} }
\newcommand{\olapl}{ \hat{\bm{\nabla}}^2 }
\newcommand{\osig}{ \hat{\sigma} }
\newcommand{\osigv}{ \hat{\vsigma} }
\newcommand{\oepsv}{ \hat{\vepsilon} }
\newcommand{\obv}{\operatorvector{b}}
\newcommand{\okv}{\operatorvector{k}}
\newcommand{\orv}{\operatorvector{r}}
\newcommand{\onv}{\operatorvector{n}}
\newcommand{\opv}{\operatorvector{p}}
\newcommand{\osv}{\operatorvector{s}}
\newcommand{\oxv}{\operatorvector{x}}
\newcommand{\oyv}{\operatorvector{y}}
\newcommand{\ozv}{\operatorvector{z}}
\newcommand{\oAv}{\operatorvector{A}}
\newcommand{\oBv}{\operatorvector{B}}
\newcommand{\oEv}{\operatorvector{E}}
\newcommand{\oHv}{\operatorvector{H}}
\newcommand{\oJv}{\operatorvector{J}}
\newcommand{\oLv}{\operatorvector{L}}
\newcommand{\oPv}{\operatorvector{P}}
\newcommand{\oSv}{\operatorvector{S}}
\newcommand{\omuv}{\hat{\vmu}}

\newcommand{\obd}{\ob^{\dagger}}
\newcommand{\oad}{\oa^{\dagger}}

\newcommand{\hvp}{\hat{\textbf{p}}}
\newcommand{\hvq}{\hat{\textbf{q}}}
\newcommand{\hvr}{\hat{\textbf{r}}}
\newcommand{\hvu}{\hat{\textbf{u}}}
\newcommand{\hvx}{\hat{\textbf{x}}}
\newcommand{\hvy}{\hat{\textbf{y}}}
\newcommand{\hvz}{\hat{\textbf{z}}}
\newcommand{\hvtheta}{\hat{\bm{\theta}}}
\newcommand{\hvphi}{\hat{\bm{\phi}}}

\newcommand{\tA}{\tilde{A}}
\newcommand{\tB}{\tilde{B}}
\newcommand{\tC}{\tilde{C}}
\newcommand{\tE}{\tilde{E}}
\newcommand{\tP}{\tilde{P}}
\newcommand{\tvA}{\tilde{\vA}}
\newcommand{\tvB}{\tilde{\vB}}
\newcommand{\tvC}{\tilde{\vC}}
\newcommand{\tvE}{\tilde{\vE}}
\newcommand{\tvP}{\tilde{\vP}}


\newcommand{\vEo}{\vE_{\omega}}
\newcommand{\vEto}{\vE_{2 \omega}}
\newcommand{\vPo}{\vP_{\omega}}
\newcommand{\vPto}{\vP_{2 \omega}}

\newcommand{\schrodinger}{Schr\"odinger }
\newcommand{\schrodingers}{Schr\"odinger's }

\newcommand{\Hermitte}[2]{\textrm{H}_{#1}\pa{#2}}

\newcommand{\unit}[1]{~\textrm{#1}}
\newcommand{\units}[1]{\unit{#1}}
\newcommand{\unite}[1]{\unit{#1}}
\newcommand{\unites}[1]{\unit{#1}}
\newcommand{\se}{\hspace{3pt}}                          % Espace pour les équations
\newcommand{\su}{\hspace{3pt}}                          % Espace pour les unités

\newcommand{\real}[1]{ \mathcal{R}e\pa{#1} }
\newcommand{\imag}[1]{ \mathcal{Im}\pa{#1} }

\newcommand{\pa}[1]{\left( #1 \right)}
\newcommand{\cro}[1]{\left[ #1 \right]}
\newcommand{\cbraket}[1]{\left\{ #1 \right\}}
\newcommand{\mean}[1]{\left< #1 \right>}

\newcommand{\ft}[1]{\textrm{FT}\pa{#1}}
\newcommand{\fti}[1]{\textrm{FT}^{-1}\pa{#1}}

\newcommand{\chiun}{\chi^{(1)}}
\newcommand{\chideux}{\chi^{(2)}}
\newcommand{\chitrois}{\chi^{(3)}}

\newcommand{\abs}[1]{\left| #1 \right|}
\newcommand{\trace}[1]{\textrm{tr}\pa{ #1 }}
\newcommand{\tr}[1]{\trace{#1}}

\newcommand{\identity}{\textbf{1}}

% \newcommand{\bra}[1]{\left. \langle #1 \right|}
% \newcommand{\ket}[1]{\left| #1 \rangle \right.}
% \newcommand{\braket}[3]{\bra{#1} #2 \ket{#3}}
% \newcommand{\crossp}[2]{\langle #1 | #2 \rangle}
\newcommand{\bra}[1]{\left< #1 \right|}
\newcommand{\ket}[1]{\left| #1 \right>}
\newcommand{\braket}[3]{\left< #1 \left|\left. #2  \right|\right. #3 \right>}
\newcommand{\crossp}[2]{\left< #1 \left| #2 \right.\right>}
\newcommand{\matrixelem}[3]{\braket{#1}{#2}{#3}}
\newcommand{\expectation}[2]{\braket{#1}{#2}{#1}}
\newcommand{\expectationsmall}[1]{\left< #1 \right>}

% \newcommand{\cqfd}{\begin{flushright}$\Box$\end{flushright}}
% Needs: \usepackage{latexsym}
\newcommand{\cqfd}[1]{\begin{tabular*}{\textwidth}{@{\extracolsep{\fill}}lr}#1&$\Box$\end{tabular*}}

\newcommand{\citeneeded}{\textbf{[Citation Needed]}}
\newcommand{\citen}[1]{\textbf{[#1]}}

\newcommand{\emptyfig}[2]{
\begin{figure}
    \begin{center}
        \framebox[0.50\columnwidth]{\rule{0pt}{150pt}}
    \end{center}
    \caption{\label{#2}#1}
\end{figure}
}

% http://tex.stackexchange.com/questions/103608/how-to-force-latex-not-to-break-the-line-after-a-hyphen
\newcommand{\optprefix}[2]{% optional prefix
  #1\mbox{-}\nobreak\hspace{0pt}#2
}
% Version with a ``%'' at the end to prevent a newline created even when ``~'' is used after
\newcommand{\optprefixp}[2]{% optional prefix
  #1\mbox{-}\nobreak\hspace{0pt}#2%
}

\newcommand{\xray}{\optprefix{X}{ray}}
\newcommand{\xrays}{\optprefix{X}{rays}}
\newcommand{\xrayp}{\optprefixp{X}{ray}}
\newcommand{\xraysp}{\optprefixp{X}{rays}}

%%%%%%%%%%%%%%%%%%%%%%%%%%%%%%%%%%%%%%%%%%%%%%%%%%%%%%%%%%%%%%%%%%%%%%%%%%%%%%%%%%%%
%%%%%%%%%%%%%%%%%%%%%%%%%%%%%%%%%%%%%%%%%%%%%%%%%%%%%%%%%%%%%%%%%%%%%%%%%%%%%%%%%%%%


