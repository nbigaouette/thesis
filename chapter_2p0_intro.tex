A large selection of methods exist to study laser-cluster interaction,
differentiating themselves through the amount of approximation taken.

Exact solution of the quantum mechanical system is, most of the time,
intractable. Theoretical investigations thus require some degree of
approximation, a compromise between feasibility and exactitude. On one end of
the spectrum, the most general methods are solving the
Time-Dependent \schrodinger Equation (TDSE) directly and Quantum Monte-Carlo
(QMC) methods\cite{Nightingale1998}. Unfortunately, these methods can only be applied to the simplest
systems of small numbers of electrons; clusters cannot be studied using these
methods.

Larger systems can be studied using \textit{ab initio} methods (``from
first principles'') which covers a wide range of techniques. In this class of
methods one can find \textit{Hartree-Fock} (HF) methods which consist on
approximating the ground state wavefunction by a single Slater
determinant\cite{Laaksonen1986,Schafer2009}.
In HF methods, instantaneous electron-electron Coulomb repulsion is not
included directly in the system's Hamiltonian. Instead, only the average field
resulting from other electrons is used, giving the often used name of
\textit{self-consistent field} methods. Other \textit{ab
initio} methods are the \textit{Post Hartree-Fock} methods where electron
correlation is added\cite{Cramer2004}. An example is the \textit{Configuration Interaction} (CI)
method. Because of their great accuracy, these methods are restricted to
relatively small systems, generally less then 10 atoms. Full \textit{ab initio}
treatment of clusters is also not possible.

Larger systems require
more approximations. \textit{Density Functional Theory}
is an often used method for cluster studies requiring quantum aspects, with
either quantum or semiclassical propagation\cite{Schafer2009,Fennel2010}.
It starts by formulating an
expression for the total energy of electrons and ions and derives static and
dynamic equations from it. All approximations are done in the selection of this
(energy) functional. The upper limit on these methods is of practical reasons,
mainly computational power available. On the other side, because the chosen
functional approximate the underlying quantum system, specific quantum effects
might not be included, for example shell effects or tunnelling are neglected.

Because DFT methods are mean field in nature, they cannot account for the large
field fluctuations seen in strong field cluster dynamics.

Continuing on the spectrum of methods, classical methods likes Molecular
Dynamics (MD) solves Newton
equations of motion with any kind of forces between particles\cite{Skeel1998}.
When used in
laser-cluster simulations, MD generally uses the instantaneous electrostatic
(Coulomb) force between charged particles. MD codes have the advantage of being
straightforward to implement and use, give macroscopic properties easily and allow a
relatively large amount of particles to be simulated, while still allowing
specific quantum treatment when necessary (for ionization events for example). For
these reasons it is the method of choice for use in the present work.

A limitation of MD is that it cannot account for field propagation effects. For
large clusters of more than tens of thousands of atoms, these effects can become
important. PIC methods, where particles interact with a grid which propagates the
electromagnetic field (through Maxwell's equations) that other particles see
intrinsically describe these retardation effects. PIC can be successfully used
to model large clusters, but they tend to be more complex and so lack the
simplicity of MD. Additionally, because they treat particles to be of the same
size as the underlying grid, they do not describe close range interactions.
Nevertheless, Varin \textit{et al.} developed a new method that add microscopic
corrections to PIC close range interactions\cite{Varin2012}. With these microscopic
corrections, collisions between particles are described similarly to MD, insuring
that the a better description of the dynamics. This MPIC method has great prospects
as not only can it describes what MD does but can go further by including the
electromagnetic field propagation and its effects. Additionally, its scaling
allows it to describe larger clusters and it can be parallelized more easily
than MD. For smaller systems though, the complexity of PIC and MPIC and their
grid's overhead gives MD a clear advantage.

Some refinements to the MD algorithm allow it to be used for larger systems
while still keeping the same underlying code structure. For example, organizing
particles in a hierarchical tree\cite{Barnes1986,Gibbon2002} can speed up the force
calculation from a $O\pa{N^2}$ scaling to $O\pa{N \log{N}}$. More details on the
tree algorithm is presented in section \ref{section:intro:md:tree}.

On the end of the methods spectrum lies rate equations methods where a time
dependence for parameters is written down and solved. Too many approximations
are made by rate equations for use in the current work. For example, they often
assume an infinite bulk. They also cannot describe the large variation in field
and charge present in the clusters.

For a detailed review of the different methods, see Ref. \cite{Fennel2010}.

In the following section, the tools and their implementation used in the present
work will be presented. Unless stated otherwise, all implementations are original
and were done from scratch, except the Hartree-Fock-based Cowan code\cite{CowanCode} used to
in the calculation of various required cross-sections.


