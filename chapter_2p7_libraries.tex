\subsection{Libraries}
\label{section:tools:libraries}

Additionally to the MD and QFDTD packages, eight libraries were developed to
support some generic features shared between the two.

\subsubsection{timing.git} \label{section:tools:libraries:timing}

This library includes timing routines used to measure running time, estimated
time of arrival (ETA), code profiling, etc. It consists of 1,300 lines of C++
code and can be found here:
\url{https://gitlab.cphoton.science.uottawa.ca/nbigaouette/timing}.


\subsubsection{stdcout.git} \label{section:tools:libraries:stdcout}

Add logging features allowing saving the output of any code execution to a
(compressed or not) log file consisting of 500 lines of C++ code.
Can be found here:\\
\url{https://gitlab.cphoton.science.uottawa.ca/nbigaouette/stdcout}.


\subsubsection{prng.git} \label{section:tools:libraries:prng}

A pseudo-random number generator (PRNG) library. Wrapper around
dSFMT\cite{prng2009}, a SIMD-oriented Fast Mersenne Twister implementation.
Defines easy to use PRNG functions, distributions, seed, etc. Keeps track of
the seed and how many times pseudo-random numbers were generated, allowing
to replay the series in case of simulation reloading. 500 lines of C++ code
(not counting dSFMT) and found here:\\
\url{https://gitlab.cphoton.science.uottawa.ca/nbigaouette/prng}


\subsubsection{memory.git} \label{section:tools:libraries:memory}

Some wrappers around malloc() and calloc() that will automatically check that
memory allocation succeeded. Will also keep track of the amount of
memory allocated and allow setting a maximum value, preventing
over-allocation due to bugs or user errors. Can also return the binary
representation of a number as a string for easier debugging. 1,200 lines of
C++ code. Available here:\\
\url{https://gitlab.cphoton.science.uottawa.ca/nbigaouette/memory}


\subsubsection{io.git} \label{section:tools:libraries:io}

Input and Output library. Includes wrappers around TinyXML library\cite{tinyxml}
for reading simulation input files in XML format,
wrappers around NetCDF\cite{netcdf} for self-contained
output files, used for both post-processing and simulation snapshots.
2,950 lines of C++ code (not counting TinyXML). Available here:\\
\url{https://gitlab.cphoton.science.uottawa.ca/nbigaouette/io}


\subsubsection{libpotentials.git} \label{section:tools:libraries:libpotentials}

Functions implementing and abstracting different potential shapes as
described in section \ref{section:intro:md:potentials}. 4,300 lines of C++ code.
Available here:\\
\url{https://gitlab.cphoton.science.uottawa.ca/nbigaouette/libpotentials}


\subsubsection{oclutils.git} \label{section:tools:libraries:oclutils}

Library to ease the use of OpenCL devices. Allows listing and selecting the
best GPU available on a workstation and locking it to prevent
other simulations from using it. Contains an array abstraction to ease the
transfer of data from the host (main memory) to device (GPU memory) and
vice-versa. Also includes a SHA512 checksum check to validate memory
and detect any issue during transfers. 3,300 lines of C++ code.
Available here:
\url{https://gitlab.cphoton.science.uottawa.ca/nbigaouette/oclutils}


\subsubsection{get\_libraries.git} \label{section:tools:libraries:getlibraries}

Simple script that will download all the required libraries or
update them to the lasted version from git, compile them and install them in the
user's directory. 240 lines of bash code.
Available here:\\
\url{https://gitlab.cphoton.science.uottawa.ca/nbigaouette/get_libraries}



\subsubsection{Ionization library}
\label{section:tools:libraries:ionization}

All ionization processes described in section
\ref{section:intro:mechanisms} and consisting of 18,000 lines of C++ code.
The library is available here:\\
\url{https://gitlab.cphoton.science.uottawa.ca/nbigaouette/ionization}

The ionization library is a large part of the work as can be seen by the amount
of line of code. Each ionization processes of section
\ref{section:intro:mechanisms} have a function that, when called, will iterate
over the list of ions and calculate if the ionization takes place or not. In the
case of collisional processes (impact ionization and ACI), the iteration is over
electrons instead. The MD stores each electrons' nearest neighbour in a flag,
allowing the ionization library to calculate the required values for ionization.

To keep the library generic (so it can be used with other simulation packages,
not just the MD developed here), the library does not explicitly create
electrons (or recombine them). It instead calls functions that \textit{must}
be defined in the simulation package for the exact values to set.


